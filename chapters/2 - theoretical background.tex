\providecommand{\main}{..}
\documentclass[\main/master.tex]{subfiles}
\begin{document}
\chapter{theoretical background}\label{chp:example-2}


\section{Torsional Pendulum}
\subsection{Torsional Pendulum}
Torsion Pendulum is an oscillator made of mass hung by a string from a fixed point so it could swing free. When the mass is displaced from equilibrium angle, the pedulum is having a liniar restoring tourqe, caused by the twisted string. The restoring tourqe is rotating the mass back to equilibrium position.


\begin{figure}[htbp]
	\centering
	\fbox{\includegraphics[scale=1.2]{\main/images/2 - theoretical background/torsion_pendulum.JPG}}
	\caption[pendulum]{a torsion pendulum}
	\label{fig:torsion_pendulum}
\end{figure}

Assuming a mass with a profile which could be estimated to a long thin rod. When displaced from equilibrium, there are two sources of torque; the displacing tourqe and a liniar restoring tourqe at opposite direction. At equilibrium, when balance stabilizes in a specific angle, the two sources of torque cancel out. With small angles the pendulum obeys Hooke’s law, when $L$ is the rod length.
\begin{equation}
\tau = -\kappa\theta = LF    \label{eqn:Hooke_law}
\end{equation}
The moment of innertia of a rod oscillating around a pivot.
\begin{equation}
I = \frac{ML^2}{12}     \label{eqn:moment_innertia}
\end{equation}  


The string torsion coefficient $\kappa $ could be estimated, assuming homogeneousity of a circular string. Where G is the material shear modulus, j is the circular second moment of area ($Jzz$) and $h$ is the string length.
\begin{equation}
\kappa = \frac{GJ}{h} = \frac{G}{h} \frac{\pi D^4}{32}    \label{eqn:torsion_coefficient}
\end{equation}
Note  $\kappa$ could also be found empirically by finding  the  oscillations period.
 


\subsection{Simple Harmonic Oscillator}
Harmonic oscillator is a second order system that, when displaced, experiences a liniar restoring force proportional to the displacement from equilibrium. Torsional pendulum is an angular harmonic oscillator (equivalent to the liniar case, instead of velosity and force, angular velosity and tourqe).
\par

If the restoring tourqe is the only tourqe acting, the oscillator is not driven or damped, called a simple harmonic oscillator. The tourqe is causing sinusoidal oscillations (simple harmonic motion) around the equilibrium angle. The time period and tourqe are determined by the physical constants of the pendulum. 
\begin{equation}
\tau = -\kappa\cdot\theta  = I\cdot\ddot{\theta}   \label{eqn:undamped_motion_equation}
\end{equation}
\begin{equation}
\theta(t) = \theta_{max}cos(\omega_0 t )    \label{eqn:undamped_motion_equation}
\end{equation}
\begin{equation}
\omega_0  = \frac{2\pi}{T} = \sqrt{\frac{\kappa}{I}}   \label{eqn:undamped_motion_equation}
\end{equation}

\subsection{Damped Oscillator}
If the system is also having damping (friction) which is proportional to the velocity, the system is called damped oscillator.


\begin{equation}
\tau_{drag} = -b\cdot\dot{\theta}   \label{eqn:friction_tourqe}
\end{equation} 
\begin{equation}
\tau = -\kappa\cdot\theta - b\dot{\theta}  = I\cdot\ddot{\theta}   \label{eqn:damped_motion_equation}
\end{equation} 
\begin{equation}
\ddot{\theta} + 2\xi\omega_0\dot{\theta} + \omega_0^2\theta = 0   \label{eqn:damped_motion_equation}
\end{equation}

The system damping ratio $\xi$ is the ratio between critical damping and the actual damping. The parameter determines the type of damping the system has, and the damping time of the system $\tau$. The damping ratio is a function of the damping coefficient $b$. Sytem could either be undamped, overdamped, critically damped or underdamped.
\begin{equation}
\theta(t) = Ae^{-\omega_0 t(\xi+\sqrt{\xi^2-1})} + Be^{-\omega_0 t(\xi-\sqrt{\xi^2-1})}    \label{eqn:damped_motion_equation}
\end{equation} 
\begin{equation}
\xi = \frac{b}{2\sqrt{I\kappa}} = \frac{actual}{critical}  \label{eqn:damped_motion_equation}
\end{equation}
\begin{equation}
\tau = \frac{1}{\xi\omega_0} = \frac{1}{\frac{b}{2\sqrt{I\kappa}}\sqrt{\frac{\kappa}{I}} }= \frac{2I}{b}  \label{eqn:damping_time}
\end{equation}
\begin{equation}
\theta(t) = Ae^{-\frac{t}{\tau}\cdot(1+\sqrt{1-\frac{1}{\xi^2}})} + Be^{-\frac{t}{\tau}\cdot(1-\sqrt{1-\frac{1}{\xi^2}})}    \label{eqn:damped_motion_equation}
\end{equation}




Undamped ($\xi = 0$); there is no friction and damping $b = 0$, harmonic oscillations without decay.
\begin{equation}
\theta(t) = \theta_{max}cos(\omega_0 t )    \label{eqn:undamped_motion_equation}
\end{equation}
Overdamped ($\xi > 1$); due to high friction, the system cannot oscillate and decays exponentialy to equilibrium position.
\begin{equation}
\theta(t) = Ae^{-\frac{t}{\tau}\cdot(1+\sqrt{1-\frac{1}{\xi^2}})} + Be^{-\frac{t}{\tau}\cdot(1-\sqrt{1-\frac{1}{\xi^2}})}    \label{eqn:overdamped_motion_equation}
\end{equation}
Critically damped ($\xi = 1$); due to high friction, the system cannot oscillate and decays exponentialy to equilibrium position. For a fixed $I, \kappa$, choosing $b$ to be at the critical damping value gives
the fastest return to equilibrium position. Although there is overshoot this is often a desirable property.

\begin{equation}
\theta(t) = \theta_{max}\cdot e^{-\frac{t}{\tau}}     \label{eqn:underdamped_motion_equation}
\end{equation}
 underdamped ($\xi < 1$); amplitude decreases in time due to the friction while oscillating with a lower frequency due to the damping. When the damping coefficient is small enough, the frequency change is neglected.
\begin{equation}
\theta(t) = \theta_{max}\cdot e^{-\frac{t}{\tau}}cos(\sqrt{1-\xi^2}\omega_0 t ) =  \theta_{max}\cdot e^{-\frac{t}{\tau}}cos(\omega t )    \label{eqn:underdamped_motion_equation}
\end{equation}
\begin{equation}
\omega = \omega_0\sqrt{1-\xi^2}\approx\omega_0    \label{eqn:underdamped_frequency}
\end{equation}

\begin{figure}[htbp]
	\centering
	\fbox{\includegraphics[scale=0.2]{\main/images/2 - theoretical background/damping.png}}
	\caption[damped]{damped oscillators; wikipedia}
	\label{fig:damped_oscillators}
\end{figure}


\subsection{Overshoot}
Overshoot is when output signal or function exceeds the target value. The response signal is not accurate compare to target. In controll theory there are two wanted conflicting properties; an accurate response (small overshoot), and small risetime (fast response). 
\par
Overshoot is usually measured in percentage overshoot (PO). For second order systems, such as damped oscillators PO is a function of the damping ratio $\xi$. 


\begin{equation}
PO = 100\cdot e ^{\frac{-\xi\pi}{\sqrt{1-\xi^2}}} = \frac{output-target}{target}   \label{eqn:percentage_overshoot}
\end{equation}
 



\iffalse
https://ocw.mit.edu/courses/mathematics/18-03sc-differential-equations-fall-2011/unit-ii-second-order-constant-coefficient-linear-equations/damped-harmonic-oscillators/MIT18_03SCF11_s13_2text.pdf

https://www.sciencedirect.com/topics/engineering/underdamped-system#:~:text=When%20the%20damping%20ratio%20is%20between%200%20and%201%20(i.e.,is%20an%20exponential%20decay%20line.
\fi

\subsection{Driven Oscillator}
If the damped oscillator system is further affected by an external time-dependent tourqe $\tau(t)$,  the system is called driven oscillator.
\begin{equation}
\tau(t) -\kappa\cdot\theta - b\dot{\theta}  = I\cdot\ddot{\theta}   \label{eqn:driven_motion_equation}
\end{equation} 
\begin{equation}
\ddot{\theta} + 2\xi\omega_0\dot{\theta} + \omega_0^2\theta = \frac{\tau(t)}{I}   \label{eqn:damped_motion_equation}
\end{equation}




\section{PID Controller}
\subsection{PID Controller Feedback}
proportional–integral–derivative controller (PID controller) is a feedback based control system. The control loop is used for time continuouse control of a process, so the process output (measured process variable) would be close to a defined set point.
\par
PID controller continuously calculates error value, which is the distance of the measured process variable from the defined set point. The feedback applies an external correction to the process, called control variable. The control variable is based on a proportional, integral and derivative gain to the error. Proper tuning of a PID enables an accurate automated correction to a controled process. The PID concept is used widely in applications requiring accurate automated control.



\par
The error and the error's integral and derivate are calculated continuously. The control is having three tuning parameters; $P, I, D$, the feed back correction is modulated by the tuning parameters value. Each parameter is assigned to a gain (proportional, integral and derivative). The response (control variable) is a weighted sum of the control terms. Over time the controller attempts to minimize the error $e(t)$ by adjusting the control variable $u(t)$, which is the PID output.
\par
\begin{figure}[htbp]
	\centering
	\fbox{\includegraphics[scale=0.2]{\main/images/2 - theoretical background/PID.png}}
	\caption[PID]{PID controller feedback loop; from wikipedia}
	\label{fig:PID_scheme}
\end{figure}
\begin{equation}
u(t) = K_Pe(t)+K_I\int_{0}^{t}e(t)+K_D\frac{de(t)}{dt}   \label{eqn:PID_eq}
\end{equation}


P is proportional to the current error value $e(t)$. When error is large and positive, proportional output would be proportionately large and positive.
\par
I is proportional to past error value's integration. The response of the cumulative value could eliminate residual errors, such as signal offset.
\par
D is proportional to error current change rate, by calculating the derivative. The response of the change rate could eliminate more rapid changes.
\par
The optimal control function is achieved by balancing the responses. The tuning constants depend on the characteristics of the specific procces. The response must be tuned for each control application.  

\subsection{Damped Oscillator}
PID controller can continuously calculates error value of an oscillator, to damp it to zero. If the defined set point is set to zero, the error is the measured process variable. The PID acts as friction, gradually working when the ossicilations are at the maximum speed to slow them down, and remove the tourqe energy.
\begin{equation}
e(t) = \theta(t) = \theta   \label{eqn:error}
\end{equation}
\begin{equation}
\tau_{PID} = -\gamma\cdot\dot{\theta}   \label{eqn:friction_tourqe}
\end{equation}
The sytem is a damped oscillator, with an external force correction, which inserts tourqe to the oscillator.
\begin{equation}
\kappa\cdot\theta - \gamma\cdot\dot{\theta}  + I\cdot\ddot{\theta} = 0   \label{eqn:damped__pid_motion_equation}
\end{equation}
\begin{equation}
\gamma\dot{\theta}  = Fr   \label{eqn:damped__pid_motion_equation}
\end{equation}
\begin{equation}
\dot{\theta} = \omega_0\theta_{max}sin(\omega_0 t +\phi)    \label{eqn:undamped_motion_equation}
\end{equation}
\begin{equation}
\dot{\theta}_{max} = \omega_0\theta_{max} = \theta_{max}\cdot\frac{2\pi}{T}    \label{eqn:undamped_motion_equation}
\end{equation}
\begin{equation}
\gamma  = \frac{F_{max}r}{\dot{\theta}_{max}} =\frac{F_{max}rT}{\theta_{max}2\pi}    \label{eqn:damped_pid_motion_equation}
\end{equation}
\begin{equation}
\tau =  \frac{2I}{\gamma}  \label{eqn:damping_time}
\end{equation}
The PID damping $\gamma$ and damping time $\tau$ depends on the initial oscillations angle. If the angle is too large compare to the inserted force, the system is extremly underdamped and the PID affect is neglected. The damping time would be infinite and the system would keep on ossicilating.

\subsection{Control Stability}
Using PID control does not guarantee optimal control or stability. The control system is aiming to achieve critical damping of the process. Well tuned control would a reach the desired set point fast and accurate, and also apply over time the nesseary corrections to resist external forces trying to move variable from the set point.
\par
The controller response is its response to error. How much does the sytem overshoots a setpoint and the system oscillations. When controller gains are too high, instead of critical damping there is overdamping causing overshoot, due to the high gain the overshoot response overshoots to the other side, causing the system to be driven.





\section{Gravity Measurement}

\subsection{Gravitational Field}
Newton law of universal gravitation states that every point mass attracts every other point mass in the universe.
\begin{equation}
\overrightarrow{F}(r) = \frac{GMm}{r^2}\hat{r}    \label{eqn:gravitation_force}
\end{equation} 
The force is acting along the intersecting line, proportional to the product of the masses and inverse to the square of the distance between the centers. Since it is inverse to the distance square means the force is very weak. For example the force between two cubes 1 kg each 1 meter away would be $\overrightarrow{F} = 6.67\cdot10^{-11} [N]$.
\par
Gravitational field of a mass is a vector field consisting at every point a vector pointing directly towards the particle. The magnitude of the field at every point is calculated by applying the universal law, the force per unit mass. 
\begin{equation}
\overrightarrow{G}(r) = \frac{\overrightarrow{F}(r)}{m} = \frac{GM}{r^2}\hat{r}    \label{eqn:gravitation_field}
\end{equation}
The field caused by a mass at a specific point is measured by measuring the gravitational force. The gravitational force caused by a small known mass test mass $m$ relative to the mass $M$. Test mass $m$ much smaller than mass $M$ ensures that there is a negligible influence on the behavior of $M$.  



\subsection{Cavendish Experiment}
The Cavendish experiment, first performed in the 17th century, was the first experiment to measure the gravitational force between masses. The apparatus which is constructed by a torsional pendulum is still used to measure accurately gravitational forces. Assuming no friction or other damping force (simple harmonic oscillator) when a mass is interduced, there are two sources of torque in the system; tourqe by the mass gravitational force, and restoring tourqe caused by the wire torsion. Since gravity is a weak force the torsional pendulum obeys Hooke’s law (small angles). At equilibrium, when balance has been stabilized at an angle $\theta$, these two tourqes are canceled out.
\begin{equation}
\overrightarrow{\tau} = \kappa\theta = LF = L\frac{GmM}{r^2}    \label{eqn:gravitation_tourqe}
\end{equation}


\subsection{Gravity Sensing}
The field at a specific point is the super position of fields point value, of fields caused by any mass in space. Measuring accurately the gravitational field caused by a mass, with a known test mass, the mass weight or distance could be estimated. In order to filter out the other gravitational fields, caused by masses around, the measurement is measuring the difference from a baseline state.
\par
Usually the measurement is with a frequency. The test mass is moving in different frequencies (not moving is zero frequency). The masses around are not moving, and are filtered out.
\par
The angle caused by the gravitational field is measured at each of those frequencies, measuring the energy spectral density $[\frac{rad}{\sqrt{Hz}}]$.
In the measurement system different noises have different frequencies. By integrating over long time measurement noises are reduced. The signal to noise ratio (SNR) is the limiting factor for the sensetivity of a measuring system. In higher frequencies integration is faster, causing less noises and higher SNR.
The SNR dependency on frequency cause difficuties measuring gravitational fields at low frequencies.

\subsection{shot noise}
When measurement is an optical, such as measuring angle displacement with a laser, the ground SNR that could be achieved is limited by the system shot noise. Shot noise is a fundamental quantum physical phenomenon. Power fluctuations due to fluctuations in the number of photons. 
\begin{equation}
SNR = \frac{N}{\sqrt{N}} = \sqrt{N}    \label{eqn:shot_noise}
\end{equation}
While other system noises are higher than the shot noise limit, reducing noises and time integration would provide higher accuracy and better mesurement results. 







\section{Radiation pressure}
Radiation pressure is pressure on the surface due to momentum exchange with electromagnetic field, including momentum of light. The light is at any wavelength which is absorbed, reflected, or emitted. The pressure is causing force on the surface, although the force is usually insignificant.
  
\par
The radiation pressure force depends on the angle of surface compare to electromagnetic field, surface intesity reflectance and absorbance, and the power of light hitting the surface $\Theta_i$ (radiant flux, measured in watts). There is coupling efficiency $\eta$ due to the light losses while passing through light transmitters (such as light guide or fiber), and size difference between the light spot size and the target size.
\begin{equation}
P_{incident} = \frac{\frac{\Theta_e}{A}cos^2(\alpha)}{c} = \frac{\eta\cdot \Theta_{source}\cdot cos^2(\alpha)}{{A\cdot c}} \label{eqn:energy-mass-equivalence-relation}
\end{equation}
The incident radiation pressure, when the angle $\alpha$ is neglected, $A$ is the surface area and $c$ the speed of light.

\begin{equation}
F = P_{total}\cdot A = (P_{incident}+P_{emitted})\cdot A = 2\cdot P_{incident}\cdot A\label{eqn:energy-mass-equivalence-relation}
\end{equation}
\begin{equation}
F = \frac{2\eta\cdot\Theta_{source}}{{A\cdot c}}\cdot A = \frac{2\eta\cdot\Theta_{source}}{{c}} \label{eqn:energy-mass-equivalence-relation}
\end{equation}

%\begin{equation}
%\Theta_{source} = Q_e\cdot t = n\cdot\frac{\hslash c}{\lambda}\cdot t  \label%{eqn:energy-mass-equivalence-relation}
%\end{equation}
The radiation force assuming light field direction perpendicular to surface, and angle neglected. Light source with 1 watt maximum power and 40\% efficiency could produce nano newton forces.  

 

\section{Vacuum Theory}

\subsection{Leak Rate}
At any gas system, some gas would slowly leak over time and increase the pressure if not pumped out. The leak rate $Q_L$ is not a function of time, and resuls in a sustained increase of pressure P over time.
\begin{equation}
Q_L = \frac{\Delta P\cdot V}{\Delta t}  \label{eqn:energy-mass-equivalence-relation}
\end{equation}
The leak rate could also prevant the system from reaching initial low pressure (at some point the leakage would be equal to the pumping rate).
\par
There is also diffusion rate of gas molecules, such as helium, which is insignifant when pressure is above $10^{-9}$ [Torr] (less than high vacuum). 

\subsection{Outgassing}
Outgassing is the desorption of gas molecules in vacuum (primarily water)  that were adsorbed or absorbed in the material. The outgassing occures over time and increases the pressure farther more. At low pressures, there are more gas molecules adsorbed on the chamber surface than floating in the chamber. The total surface area is more important than the volume for reaching low pressure. 
\par
The desorption rate $Q_{des}$ of the surfaces produces a gas yield that declines over time, dependent on the desorption density which is area specific. It could be assumed that after a given time $t>t_0$ the increase is liniar over time, typicaly $t_0$ assumed to be one hour.
\begin{equation}
Q_{des} = q_{des}\cdot A\cdot\frac{t_0}{t}  \label{eqn:energy-mass-equivalence-relation}
\end{equation}
Outgassing is minimized by selection of low vapor pressures materials such as stainless steel and glass. Since water is a significant source of outgassing, it is usually minimize by baking the chamber in high tempertures while the pump is running.
\subsection{Vacuum Chambers}
Vaccuum chamber is a metalic chamber with a vacuum pump which has a pressure lower than the atmospheric pressure (which is 760 [Torr]). Quality of vacuum is divided to ranges; low vacuum, medium vacuum, high vacuum and ultra high vacuum. The main limitations to vacuum quality maintenance are leakage from the outside through the chamber and outgassing inside the chamber.
\par
Leaks and outgassing both increases the pressure (lowering the vacuum) at a constant rate over time. Usually, in order to keep equilibrium, the leaks and outgassing overcome by having a constant pump rate (vacuum pump working).   
\begin{equation}
Q_P = Q_L + Q_{des}  \label{eqn:vacuum_equilibrium}
\end{equation}
\subsection{Faraday Cage}
Faraday cage is a cage made by a continuous covering of a conductive material which is used to block external electromagnetic fields. Faraday cage cannot prevent fields caused by electric charges placed inside.
\par
The external electrical fields causes an electric charge distribution in the conductive cage without passing inside. The cage blocks better penetration of high frequency magnetic fields, depending on the material skin depth $\delta$ which determines the shield penetrating frequencies. Electric current decays exponentially with depth through the material, the thicker the shield the better at attenuating lower frequency fields.
\begin{equation}
\delta = \frac{2\rho}{(2\pi f)(\mu_0\mu_r)}     \label{eqn:mean-free-pass}
\end{equation}
\iffalse
Where $\rho$ is the material electric resistivity, and $\mu_r$ is the relative magnetic permeability.   Vacuum chambers are usually made of a few mm thick stainless steel 316, making them faraday cages with a skin depth which could block magnetic frequencies higher than 10 KHz and reduce magnetic noise from lower frequencies.
\fi

\section{Pressure Affects}
\subsection{Flow Characteristics}
The motivation to use vacuum is to reduce brownian motion and viscous friction caused by gas particles in the medium. Gas particles fluctuating and colliding with an average free pass. 
\par
Gas pressure at the medium is propotional to the number density of particles and inverse to the mean free pass of each particle.    
\begin{equation}
P = \frac{N}{V}\cdot k_B\cdot T  \label{eqn:ideal-gasses}
\end{equation}
\begin{equation}
I = \frac{k_B\cdot T}{\sqrt{2}\cdot\pi\cdot p\cdot d_m^2}     \label{eqn:mean-free-pass}
\end{equation}
Gas particle collide many times, the mean free path is the average distance the particle could pass between two collisions with other particles. Knudsen ratio could be derived from the term, ratio which is used to characterize types of gas flow compare to the pipe diameter.
\begin{equation}
K_n = \frac{I}{d}     \label{eqn:mean-free-pass}
\end{equation}

\subsection{Viscous Friction}
The friction caused by gasses is viscous friction, which is velocity dependenant. The velocity dependence is complicated, at very low speeds, gas resistance is approximately proportional to velocity. 

\begin{equation}
F_{drag} = -b\cdot v  \label{eqn:energy-mass-equivalence-relation}
\end{equation}

At the medium vacuum regime $0.01<k_n<0.5$ the flow is called Knudsen flow. The molecules do not interact and move in straight lines between points. If the vaccum chamber is having medium vacuum or higher, classical viscous friction of objects with gas particles inside the chamber could be neglected.

\subsection{Brownian Motion}
Brownian motion is the pattern of random particles fluctuations inside a fluid, a random walk with no preferential direction of flow. This pattern happens at thermal equilibrium in a given temperature (on average there is no linear and angular momentum). 
\par
Maxwell-Boltzmann distribution of molecular speed can extract the expression of average kinetic energy for a gas particle and the total brownian motion kinetic energie.
\begin{equation}
f(v) = 4\pi(\frac{m}{2\pi kT})^{3/2}v^2exp(\frac{-mv^2}{2kT})     \label{eqn:Maxwell_Boltzmann}
\end{equation}  
\begin{equation}
<E_k>=<\frac{mv^2}{2}> = \int_{0}^{\infty}\frac{mv^2}{2}f(v)dv =  \frac{3kT}{2}    \label{eqn:avrage_kinetic}
\end{equation}
\begin{equation}
E_k=N<E_k> \frac{3}{2}NkT = \frac{3}{2}PV    \label{eqn:total_kinetic}
\end{equation}

The Brownian motion kinetic energie is proptional to the number of particles. Vacuum reduces the pressure and number of particles, which reduces the energie. Also,there is energie coupling from the envirement on the sides of the medium. Reducing the number of particles reduces the energie coupling.

\subsection{Acoustic wave}
Acoustic waves is an energy propagating through material, by adiabatic compression and decompression. One dimension acoustic wave equation, where the amplitude is acoustic pressure.
\begin{equation}
P-P_0 = p = p_0cos(\omega t -\kappa x)       \label{eqn:acoustic_pressure}
\end{equation}
\begin{equation}
I = pV      \label{eqn:acoustic_intensity}
\end{equation} 
The power carried by acoustic waves $[\frac{W}{m^2}]$, named acoustic intensity, depends on the material acoustic pressure and particle velosity. When the material pressure is reduced, the acoustic pressure is lower and less power is carried.







\end{document}







