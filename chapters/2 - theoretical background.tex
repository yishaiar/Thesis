\providecommand{\main}{..}
\documentclass[\main/master.tex]{subfiles}
\begin{document}
\chapter{theoretical background}\label{chp:example-2}
\section{devices}

\subsection{Laser}
The laser is a high power coherent light source based on stimulated emmission with narrow spectrum. Inside a cavity Electron is excited to a higher energy level, and forced by photon with the correct wavelength to be absorbed emmiting coherent photon. The coherence allows focusing to a tight spot, and the spot staying collimated over long distances. 
\par
Continuous wave (CW) laser are lasing constant output power over time. Usually the power output is stable but has oscillations due to having several longitudinal modes causing nano second scale ossicilations, output power is steady when averaged over longer periods. Also, over long periods of time lasers have slight power oscillations due to temperture changes in the envirement.
%The laser diode cavity face is rectangular, because of fabrication constraints. The rectangular face is causing cylindrical aberration, which   

\subsection{Acousto Optic Modulator}
Acousto Optic Modulator (AOM), uses the acousto-optic effect to diffract and shift the frequency of light using RF waves. An oscillating electric signal drives a piezoelectric transducer to vibrate causing RF waves, the transducer attached to a material. This causes sound waves in the material, and periodic index modulation causing a Bragg grating. Incoming light scatters off the grating and due to Bragg diffraction comes at Bragg angle.
\begin{equation}
\theta_b = sin(\theta_b)\approx \frac{\lambda}{2n\Lambda} \label{eqn:energy-mass-equivalence-relation}
\end{equation} 
$\Lambda$ is the RF wavelength, $\lambda$ is the light source wavelength. 
\par
Since all parameters constant, modulation angle is constant, and possible modulation speed is nano seconds. Giving a stable fast modulation method for CW laser.



\subsection{Light Emitting Diode (LED)}
The light emitting diode (led) is a high power long lifetime light source semiconductor based. The led is emitting light when current flows through. Led could be modulated at up to 100MHz. The emitted light is incoherent in width meaning it's hard to focus it to a point (not diffraction limited). The emitted light is incoherent in length causing wide band spectrum, although spectrum is sufficiently narrow to appear as a pure color to the human eye.
\par
The working principle is a semiconductor p-n junction with direct band gap. Meaning the lowest energy state above the band gap has the same momentum as the highest under the gap. In order to change energy state electron needs only to release energy equal to the band gap, which defines the wavelength $E_g = \frac{\hslash c}{\lambda}$. All possible energy gaps defines the spectrum, aproximatly gaussian around the wavelength. Forward voltage is applied causing electron injection and recombination with holes. The recombination is releasing energy in form of spontanious emmission photons. The modulation limitation is due to the electrons life time before recombination.
\par
The led power is proptional to the current flows through, Shockley diode equation for p-n junction defines the current relation to the diode voltage. In high enough values the power could be aproximated by liniar aproximation of the diode voltage.
\begin{equation}
P\propto e^{E/cdot V_d}\label{eqn:energy-mass-equivalence-relation}
\end{equation}

\subsection{arduino}
Arduino is an open-source microcontroller board project used for building low cost and simple digital devices and circuits. Each microcontroller contains a microprocessor, controller, serial communication interface and is equipped with digital and analog input/output pins. The microcontrollers are controlled using C and C++ programming languages, and could be operated as stand alone or connected to the computer through serial communication. 
\par
There are multiple Arduino board models, we would focus on Arduino Mega 2560. The Arduino Mega 2560 is based on the ATmega2560, an Atmel 8-bit AVR controller. Also the board has 54 digital input/output pins of which 15 could be used as PWM outputs and a 16 MHz crystal oscillator (clock). In reality the arduino doesn't have analog output, to modulate the output Pulse Width Modulation technique is used.
\par
The controller is switching on/off the signal between the full Vcc of the board and off (5-0[V]), generating a square wave. The duration of time signal on is called the pulse width. The controller is able to modulate the pulse width, and change the ratio of time signal is on compare to off. The voltage is determined by the ratio of the time voltage is on compare to time voltage off, which is called duty cycle. 100\% duty cycle means the power is always on, 0\% duty cycle power is always off and 50\% duty cycle signal is on and off in equal times. 
%\\
\par

\begin{figure}[htbp]
	\centering
	\fbox{\includegraphics[scale=0.5]{\main/images/devices/duty50.png}}
	\caption[duty cycle 50\%]{50\% duty cycle  - signal is on and off equal times}
	\label{fig:duty50}
\end{figure}
Repeating that pattern fast enough results in an analog signal as if the signal is a steady voltage. This method is able to generate signals between the full Vcc of the board and off. The signal resolution is limited by the microcontroller resolution (8-bit). Due to the fast clock the of the arduino - the PWM frequency is about 500Hz.
\par
\begin{figure}[htbp]
	\centering
	\fbox{\includegraphics[scale=0.5]{\main/images/devices/duty90.png}}
	\caption[duty cycle 90\%]{90\% of the time signal on, equivalamt to 90\% of the full Vcc signal}
	\label{fig:duty90}
\end{figure}


\par
Using this method with a led connected, the led brightness could be modulated at about 2ms speed. Since the clock is connected to all PWM pins, if they all are in the same duty cycle, they are in sync. All pins are having a changing voltage, but they all have the same voltage, alowing to connect them in series and increase the output current, and the LED power. In conclusion this is a real-time controlled, fast-modulated, high power light source. 



\subsection{Light guides}
Light guides are used to distribute light from the source to a distant area, they are made of thin filaments causing internal reflections. Light pipes are used to illuminate small areas, regardless of the spectral characteristics of the light source. They mainly depend on the cross sections of entrance and exit and the length. Making them ideal to overcoming focusing problems, such as the LED uncoherent profile.  

\section{Radiation pressure}
Radiation pressure is pressure on the surface due to momentum exchange with electromagnetic field, including momentum of light. The light is at any wavelength which is absorbed, reflected, or emitted. The pressure is causing force on the surface, although the force is usually insignificant.
  
\par
The radiation pressure force depends on the angle of surface compare to electromagnetic field, surface intesity reflectance and absorbance, and the power of light hitting the surface $\Theta_i$ (radiant flux, measured in watts). There is coupling effitiency $\eta$ due to the light passing through a light guide or fiber, and size difference between the light spot size and the target size. A is the surface area, c is the speed of light. 
\begin{equation}
P_{incident} = \frac{\frac{\Theta_e}{A}cos^2(\alpha)}{c} = \frac{\eta\cdot \Theta_{source}\cdot cos^2(\alpha)}{{A\cdot c}} \label{eqn:energy-mass-equivalence-relation}
\end{equation}

Assuming the light Direction is perpendicular to surface, and angle is neglected. Also, the light variation over the surface is negligible.

\begin{equation}
F = P_{total}\cdot A = (P_{incident}+P_{emitted})\cdot A = 2\cdot P_{incident}\cdot A\label{eqn:energy-mass-equivalence-relation}
\end{equation}
\begin{equation}
F = \frac{2\eta\cdot\Theta_{source}}{{A\cdot c}}\cdot A = \frac{2\eta\cdot\Theta_{source}}{{c}} \label{eqn:energy-mass-equivalence-relation}
\end{equation}

%\begin{equation}
%\Theta_{source} = Q_e\cdot t = n\cdot\frac{\hslash c}{\lambda}\cdot t  \label%{eqn:energy-mass-equivalence-relation}
%\end{equation}
Assuming we have a power controlled light source with 1 watt maximum power and 40\% effietency we, we could inject forces of nano newton or less.  

 

\section{vacuum theory}
A vaccuum chamber is metalic chamber with a vacuum pump which has a pressure lower than the atmospheric pressure (the atmospheric pressure is 760 [Torr]). Quality of vacuum is divided to ranges, according to the technology required to achieve it (CF, KF). Medium vacuum ($25-10^{-3}$ [Torr]), high vacuum ($10^{-3}-10^{-7}$ [Torr]) and ultra high vacuum ($10^{-7}-10^{-9}$ [Torr]). The main limitations to vacuum quality maintenance are leakage from the outside through the chamber and outgassing inside the chamber.
\par
At any gas system, some gas would slowly leak over time and increase the pressure if not pumped out. The leak rate $Q_L$ is not a function of time, and resuls in a sustained increase of pressure P over time.
\begin{equation}
Q_L = \frac{\Delta P\cdot V}{\Delta t}  \label{eqn:energy-mass-equivalence-relation}
\end{equation}
The leak rate could also prevant the system from reaching initial low pressure (at some point the leakage would be equal to the pumping rate).
There is also diffusion of gas molecules, such as helium, which is insignifant when pressure is above $10^{-9}$. 

\par
Outgassing is the desorption of gas molecules in vacuum (primarily water)  that were adsorbed or absorbed in the material. The outgassing occures over time and increases the pressure farther more. At low pressures, there are more gas molecules adsorbed on the chamber surface than floating in the chamber. The total surface area is more important than the volume for reaching low pressure. 
\par
The desorption rate $Q_{des}$ of the surfaces produces a gas yield that declines over time, dependent on the desorption density which is area specific. It could be assumed that after a given time $t>t_0$ the increase is liniar over time, typicaly $t_0$ assumed to be one hour.
\begin{equation}
Q_{des} = q_{des}\cdot A\cdot\frac{t_0}{t}  \label{eqn:energy-mass-equivalence-relation}
\end{equation}
Outgassing is minimized by selection of low vapor pressures materials such as stainless steel and glass. Since water is a significant source of outgassing, it is usually minimize by baking the chamber in high tempertures while the pump is running.
\par
In conlusion leak, and outgassing, both increases the pressure over time at a constant rate. Usually those are overcome by having the pump working, which is not possible in this experiment.
\par
The motivation to use as high vacuum as possible, in this experiment, is to reduce the friction caused by gasses. The viscous friction is velocity dependenant.
\begin{equation}
F_{drag} = -b\cdot v  \label{eqn:energy-mass-equivalence-relation}
\end{equation}


The pressure is  propotional to the number density of particles and inverse to the mean free pass of each particle.    
\begin{equation}
P = n\cdot k_B\cdot T  \label{eqn:ideal-gasses}
\end{equation}
\begin{equation}
I = \frac{k_B\cdot T}{\sqrt{2}\cdot\pi\cdot p\cdot d_m^2}     \label{eqn:mean-free-pass}
\end{equation}
Gas particle collide many times along it's way. The mean free path is the average distance the particle could pass between two collisions with other particles. Knudsen ratio could be derived from the term, ratio which is used to characterize types of gas flow compare to the pipe diameter.
\begin{equation}
K_n = \frac{I}{d}     \label{eqn:mean-free-pass}
\end{equation}
At the medium vacuum regime $0.01<k_n<0.5$ the flow is called Knudsen flow. The molecules do not interact and move in straight lines between points. Classical viscous friction of object inside the chamber could be neglected.




\section{torsion theory}

Harmonic oscillator is a system that, when displaced, experiences a liniar restoring force proportional to the displacement from equilibrium. Torsion Pendulum is an oscillator made of mass hung by a string from a fixed point so it can swing free. When the mass is displaced from equilibrium angle, the pedulum is having a liniar restoring tourqe back to equilibrium position. 
\par
If the restoring force is the only force acting, the oscillator is not driven or damped, called a simple harmonic oscillator. The force is causing sinusoidal oscillations (simple harmonic motion) around the equilibrium position. The time period and force are determined by the physical constants of the pendulum. 
\begin{equation}
\tau = -\kappa\cdot\theta  = I\cdot\ddot{\theta}   \label{eqn:undamped_motion_equation}
\end{equation}
\begin{equation}
\theta(t) = Acos(\omega_0 t +\phi)    \label{eqn:undamped_motion_equation}
\end{equation}
\begin{equation}
\omega_0  =  \sqrt{\frac{\kappa}{I}}   \label{eqn:undamped_motion_equation}
\end{equation}
If the system is also having friction $F_{drag} = -b\cdot v$, proportional to the velocity, the system is called damped oscillator. 
\begin{equation}
\tau = -\kappa\cdot\theta - b\dot{\theta}  = I\cdot\ddot{\theta}   \label{eqn:damped_motion_equation}
\end{equation} 
\begin{equation}
\ddot{\theta} + 2\xi\omega_0\dot{\theta} + \omega_0^2\theta = 0   \label{eqn:damped_motion_equation}
\end{equation}
\begin{equation}
\xi = \frac{b}{2\sqrt{I\kappa}}   \label{eqn:damped_motion_equation}
\end{equation}
Based on the friction coefficient the sytem could either be overdamped or underdamped, $\xi$ is called the damping ratio. Overdamped ($\xi > 1$); exponential decay to equilibrium position without oscillations, underdamped ($\xi < 1$); amplitude decreases in time while oscillating with a lower frequency. 
\par
If the system is also having an external time-dependent force,  the system is called driven oscillator.
\begin{equation}
\tau(t) -\kappa\cdot\theta - b\dot{\theta}  = I\cdot\ddot{\theta}   \label{eqn:driven_motion_equation}
\end{equation} 
\begin{equation}
\ddot{\theta} + 2\xi\omega_0\dot{\theta} + \omega_0^2\theta = \frac{\tau(t)}{I}   \label{eqn:damped_motion_equation}
\end{equation}
\color{blue}


\color{black}
  



\section{pid}
\section{accuracy}
\doublespacing
\hspace{5 mm} This another example chapter with a working reference as see in Chapter~\ref{chp:example-1}. There I also made an example of an equation, see Eqn.~\ref{eqn:energy-mass-equivalence-relation}. We also created an example image, see Fig.~\ref{fig:sine-wave}.
\begin{figure}[htbp]
	\centering
	\fbox{\includegraphics[scale=0.75]{\main/images/chapter_2_example/img_example_2.png}}
	\caption[Another Example Image]{Another Example Image. This image is also labeled internally so we can referenc it throughout the text.}
	\label{fig:cosine-wave}
\end{figure}
\end{document}