\providecommand{\main}{..}
\documentclass[\main/master.tex]{subfiles}
\begin{document}
\chapter{Thesis aims}\label{chp:example-2}

The Cavendish experiment, is constructed using a torsional pendulum with a front mirror. The angle displaement is causing a mirror tilt. Optical measurement of angle displaement is made with a split detector using interferometric technique.
\par\noindent
The measurement is limited by the basic energy level quantum uncertainty and measurement system shot noise.
\begin{equation}
\frac{1}{2}\kappa <\delta\theta^2>= \frac{\hslash\omega}{4}    \label{eqn:basic uncertainty}
\end{equation}
\begin{equation}
\delta\theta= \sqrt{\frac{\hslash\omega}{2\kappa}} \approx 10^{-14} [rad]    \label{eqn:basic uncertainty}
\end{equation}

\noindent
Shot noise using a split detector in an interferometric technique is even on all frequencies. Shot limit when integrating over a long time.
\begin{equation}
\delta\theta = \frac{1}{4\sqrt{2}\pi}\frac{\lambda}{L\sqrt{N}} \approx
10^{-14} [\frac{rad}{Hz}]    \label{eqn:gravitation_torque}
\end{equation}
When the gravitational force is measured using a torsional pendulum, there are noise limitation to the measuremnt sensetivity. The limitation are higher than the basic limits, meaning that reducing the noise could achieve higher sensetivity. The noises on the pendulum are unknown magnetic noises, acoustic waves and quantum uncertainty due to thermal brownian motion. Friction is also limiting the pendulum response. 
\par\noindent
Pendulum is placed inside vacuum reducing brownian motion, acoustic waves and friction, vacuum chamber is reducing magnetic noise as a faraday cage. At vacuum the brownian motion from envirement coupling is reduced.
\par\noindent
The quantum uncertainty due to brownian motion is the main noise source. The quantum uncertainty at a given temperture (assuming above basic energy level).

\begin{equation}
\delta\theta = \sqrt{\frac{KT}{\kappa}}\propto{T}  \label{eqn:radiation force}
\end{equation}
\begin{equation}
\delta\theta(t) = \delta\theta cos(\omega_0 t)   \label{eqn:pid_error}
\end{equation}
The noise reductance enables to damp and cool the pendulum temperture. Damping is made using a PID feedback loop. The feedback damping can dynamically damp the noises, without pre-knowledge of the existing noises. PID would reduce all noises, and reduce the temperature. Having an effective lower temperature would enable measuring smaller angles enlarging the SNR and the measuremnt sensetivity.
\par\noindent
When PID tuned correct damped pendulum is at critical damping and ossiclations become smaller. PID error is getting samller and damping response becomes small. 
\begin{equation}
e(t) = \delta\theta(t)   \label{eqn:pid_error}
\end{equation}
\begin{equation}
u(t) = K_Pe(t)+K_I\int_{0}^{t}e(t)+K_D\frac{de(t)}{dt}   \label{eqn:PID_eq}
\end{equation}

\noindent
Introducing a new mass adds a constant new torque to the system, which the PID is compensating with a propotional gain $K_P$. The PID response becomes mainly an inverse torque. Assuming the mass is moving at low frequencies, when integrating over short periods the angle is constant in time.
\begin{equation}
e(t) = \delta\theta(t) + \theta_g    \label{eqn:PID_measurement}
\end{equation}
\begin{equation}
u(t) = u[ \delta\theta(t)] + K_P\theta_g(t) \label{eqn:PID_measurement_eq}
\end{equation}
\begin{equation}
U = \int u(t)dt  \approx K_P\theta_g\Delta t \label{eqn:PID_measurement_eq}
\end{equation}
\begin{equation}
\tau(r) = L\frac{GmM}{r^2} =  \kappa\theta_g = \kappa\frac{U}{K_P\Delta t}      \label{eqn:pid_gravitation_torque}
\end{equation}
\begin{equation}
\kappa =\frac{8\pi^2ml^2}{T^2}    \label{eqn:empirical_torque}
\end{equation}

\noindent
The measurement of that response is fast and accurate. The measurement sytem is measuring with fast response (short integration time) and high accuracy low frequency garvity fields. 

 
 
 
 




\end{document}