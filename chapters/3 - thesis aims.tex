\providecommand{\main}{..}
\documentclass[\main/master.tex]{subfiles}
\begin{document}
\chapter{Thesis aims}\label{chapter:Thesis aims}

The Cavendish experiment, is constructed using a torsional pendulum with a front mirror. The angle displacement causes a mirror tilt. Optical measurement of angle displacement is made with a quadrant detector using interferometric technique.
\par\noindent
The measurement is limited by the basic energy level quantum uncertainty and measurement system shot noise. The basic energy level quantum uncertainty is \cite{howell2019}:
\begin{equation}
\delta\theta= \sqrt{\frac{\hslash\omega}{2\kappa}} \approx 10^{-16} [rad]    \label{eqn:basic uncertainty}
\end{equation}
\par\noindent
The corresponding shot noise limit for the interferometer is calculated as \cite{howell2019}:
\begin{equation}
\delta\theta = \frac{1}{4\sqrt{2}\pi}\frac{\lambda}{L\sqrt{N}} \approx
10^{-14} [\frac{rad}{Hz}]    \label{eqn:shot limit}
\end{equation}
Assuming above basic energy level the quantum uncertainty due to thermal energy is temperature dependent, at room temperature ($T = 300[k]$) the noise is given by \cite{howell2019}:
\begin{equation}
\delta\theta = \sqrt{\frac{k_B T}{\kappa}} \approx 4\cdot 10^{-8} [rad] \label{eqn:Brownian uncertainty 3}
\end{equation}
\par\noindent
When the gravitational force is measured using a torsional pendulum, there are noise limitations to the measurement sensitivity. The technical noise is greater than the fundamental noise, meaning that reducing the noise enables higher sensitivity. As demonstrated previously (eq.~\ref{eqn:heat conduction}, eq.~\ref{eqn:total_kinetic}, eq.~\ref{eqn:acoustic_intensity}, eq.~\ref{eqn:drag force}), Brownian motion from the environment, thermal coupling, acoustic waves and friction are pressure dependent and are considerably reduced by maintaining low pressure. Therefore, the torsional pendulum is placed inside a vacuum chamber. The noise reduction results with smaller oscillations amplitude %\delta\theta%: 
\begin{equation}
\delta\theta(t) = \delta\theta cos(\omega_0 t)   \label{eqn:pid_error}
\end{equation}
\par\noindent
The reduction of the amplitude enables to actively damp down the remaining noises and cool down the pendulum temperature. The cooling system is built using two LED light sources placed in front of each pendulum side. The led sources are modulated at high speeds using a simple Arduino microcontroller. The modulated light is causing a small external radiation torque, which damps down the oscillations.
\par\noindent
Damping is made using a PID feedback loop. The tilt angle is measured, and an equivalent feedback external source is applied. The feedback damps the noises, without pre-knowledge of the existing noises, effectively reducing the pendulum's temperature (eq.~\ref{eqn:Brownian uncertainty 3}). Having a lower effective temperature enlarges the SNR and the measurement sensitivity, which enables measuring smaller tilt angles.
\par\noindent
With PID tuned correct the damped pendulum is at critical damping and oscillations become smaller. The measured error $e(t)$ and the PID damping response $u(t)$ (eq.~\ref{eqn:PID response}) are both minimized. 
\par\noindent
Introducing a new mass adds a constant new torque to the system $\tau_g$ (eq.~\ref{eqn:net_gravitation_torque_approx}), accordingly the average tilt angle becomes $\overline{\theta}_g$ (eq.~\ref{eqn:theta average}):
\begin{equation}
\overline{\theta}_g  \approx \frac{3GT^2cos\theta sin\theta}{4\pi^2 } \cdot \frac{M}{r_0^3}   \label{eqn:theta average_2}
\end{equation}
\par\noindent
The angle is proportional to the square of the period's length $T$, the pendulum length $l$ and the masses $m$ are dropped out, which indicates that short and light weight sensors work well as larger ones, as long as their periods are the same. The resulting measured error $e(t)$:
\begin{equation}
e(t) = \delta\theta(t) + \theta_g(t)    \label{eqn:error_measurement}
\end{equation}
The PID is compensating for the new error with mainly a proportional gain $K_P$, and the response $u(t)$ becomes mainly an inverse torque:
\begin{equation}
\tau_{PID} = u(t) = u[e(t)] \approx K_P\theta_g(t) \approx \tau_{g} = \kappa \overline{\theta}_g 
\label{eqn:PID_response}
\end{equation}
Assuming the mass is oscillating at a low frequency, when integrating over short periods the angle is constant in time:
\begin{equation}
U = \int u(t) dt = K_P\overline{\theta}_g\Delta t 
\label{eqn:PID_measurement_eqn}
\end{equation}
Where $U$ is the sum of the PID response, and $\Delta t$ is the integration time. The gravitational angle is:
\begin{equation}
\overline{\theta}_g = \frac{U}{K_P\Delta t}      \label{eqn:pid_gravitation_torque}
\end{equation}
\par\noindent
According to eq.~\ref{eqn:theta average_3}, by accurate angle measurement, the gravitational field could be estimated. According to eq.~\ref{eqn:pid_gravitation_torque}, the measurement system allows fast response (short integration time) and high accuracy while measuring low frequency gravity fields. 

 
 
 
 





\end{document}