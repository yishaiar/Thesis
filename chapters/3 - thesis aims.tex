\providecommand{\main}{..}
\documentclass[\main/master.tex]{subfiles}
\begin{document}
\chapter{Thesis aims}\label{chapter:Thesis aims}

The Cavendish experiment, is constructed using a torsional pendulum with a front mirror. The angle displacement causes a mirror tilt. Optical measurement of angle displacement is made with a split detector using interferometric technique.
\par\noindent
The measurement is limited by the basic energy level quantum uncertainty and measurement system shot noise. The basic energy level quantum uncertainty:
\begin{equation}
\frac{1}{2}\kappa <\delta\theta^2>= \frac{\hslash\omega}{4}    \label{eqn:basic uncertainty}
\end{equation}
\begin{equation}
\delta\theta= \sqrt{\frac{\hslash\omega}{2\kappa}} \approx 10^{-16} [rad]    \label{eqn:basic uncertainty}
\end{equation}

\noindent
Shot noise, using a split detector in an interferometric technique, is even on all frequencies. The shot noise limit when integrating over a long time could be calculated as \cite{howell2019}:
\begin{equation}
\delta\theta = \frac{1}{4\sqrt{2}\pi}\frac{\lambda}{L\sqrt{N}} \approx
10^{-14} [\frac{rad}{Hz}]    \label{eqn:gravitation_torque}
\end{equation}
When the gravitational force is measured using a torsional pendulum, there are noise limitations to the measurement sensitivity. The limitations are higher than the basic limits, meaning that reducing the noise enables higher sensitivity. The noises on the pendulum are unknown magnetic noises, acoustic waves and quantum uncertainty due to thermal Brownian motion. Friction also limits the pendulum response. 
\par\noindent
The pendulum is placed inside a vacuum chamber, reducing Brownian motion, acoustic waves and friction. The vacuum chamber is reducing magnetic noise as a Faraday cage. In vacuum the Brownian motion from environment coupling is reduced.
\par\noindent
The quantum uncertainty due to Brownian motion is the main noise source. The quantum uncertainty at a given temperature (assuming above basic energy level) is:

\begin{equation}
\delta\theta = \sqrt{\frac{k_BT}{\kappa}}\propto{T}  \label{eqn:radiation force}
\end{equation}
\begin{equation}
\delta\theta(t) = \delta\theta cos(\omega_0 t)   \label{eqn:pid_error}
\end{equation}
\par\noindent
The noise reduction enables to damp and cool the pendulum temperature. 
\par\noindent
The cooling system is built using two led light sources placed in front of each pendulum side. The led sources are modulated at high speeds using a simple Arduino microcontroller. The modulated light is causing a small external radiation torque, which damps down the oscillations.
\par\noindent
Damping is made using a PID feedback loop. The tilt angle is measured, and an equivalent feedback external source is applied. The feedback damping can dynamically damp down the noises, without pre-knowledge of the existing noises. The PID can reduce the noises, and the effective temperature. Having a lower effective temperature enlarges the SNR and the measurement sensitivity, which enables measuring smaller tilt angles.


\par\noindent
With PID tuned correct the damped pendulum is at critical damping and oscillations become smaller. The measured error $e(t)$ and the PID damping response $u(t)$ (eq.~\ref{eqn:PID response}) are both minimized. 
\par\noindent
Introducing a new mass adds a constant new torque to the system $\tau_g$ (eq.~\ref{eqn:net_gravitation_torque_approx}), accordingly the average tilt angle becomes $\theta_g$ 
(eq.~\ref{eqn:theta average_2}). 

The PID is compensating measured error with a proportional gain $K_P$. The PID response becomes mainly an inverse torque. Assuming the mass moves at low frequencies, when integrating over short periods the angle is constant in time.
\begin{equation}
e(t) = \delta\theta(t) + \theta_g    \label{eqn:PID_measurement}
\end{equation}
\begin{equation}
u(t) = u[ \delta\theta(t)] + K_P\theta_g(t) \label{eqn:PID_measurement_eqn}
\end{equation}
\begin{equation}
U = \int u(t) dt  \approx K_P\overline{\theta}_g\Delta t \label{eqn:PID_measurement_eqn}
\end{equation}

\begin{equation}
\overline{\theta}_g \approx \frac{U}{K_P\Delta t}      \label{eqn:pid_gravitation_torque}
\end{equation}
In the previous chapter it was shown that for a torsional pendulum:

\begin{equation}
\overline{\theta} \approx \frac{3GT^2cos\theta sin\theta}{4\pi^2 } \cdot \frac{M}{r_0^3}   \label{eqn:theta average}
\end{equation}
\par\noindent
The pendulum angle is proportional to period's length $T$, pendulum length $l$ and masses $m$ are dropped out, indicating that short light weight sensors work as well as larger ones, as long as their periods are the same.
\par\noindent
The torsion string coefficient $\kappa$ could be estimated from the oscillations period. With accurate angle measurement, the gravitational field could be estimated.
\par\noindent
The measurement system allows fast response (short integration time) and high accuracy while measuring low frequency gravity fields. 

 
 
 
 





\end{document}