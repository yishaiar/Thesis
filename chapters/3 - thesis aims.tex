\providecommand{\main}{..}
\documentclass[\main/master.tex]{subfiles}
\begin{document}
\chapter{Thesis aims}\label{chapter:Thesis aims}

The Cavendish experiment, is constructed using a torsional pendulum with a front mirror. The angle displacement causes a mirror tilt. Optical measurement of angle displacement is made with a quadrant detector using interferometric technique.
\par\noindent
The measurement is limited by the basic energy level quantum uncertainty and measurement system shot noise. The basic energy level quantum uncertainty is \cite{howell2019}:
\begin{equation}
\delta\theta= \sqrt{\frac{\hslash\omega}{2\kappa}} \approx 10^{-16} [rad]    \label{eqn:basic uncertainty}
\end{equation}
\par\noindent
The corresponding shot noise limit for the interferometer is calculated as \cite{howell2019}:
\begin{equation}
\delta\theta = \frac{1}{4\sqrt{2}\pi}\frac{\lambda}{L\sqrt{N}} \approx
10^{-14} [\frac{rad}{Hz}]    \label{eqn:shot limit}
\end{equation}
Assuming above basic energy level the quantum uncertainty due to thermal energy is temperature dependent, at room temperature ($T = 300[k]$) the noise is given by \cite{howell2019}:
\begin{equation}
\delta\theta = \sqrt{\frac{k_B T}{\kappa}} \approx 4\cdot 10^{-8} [rad] \label{eqn:Brownian uncertainty 3}
\end{equation}
\par\noindent
When the gravitational force is measured using a torsional pendulum, there are noise limitations to the measurement sensitivity. The technical noise is greater than the fundamental noise, meaning that reducing the noise enables higher sensitivity. As shown previously (eq.~\ref{eqn:total Brownian energy}, eq.~\ref{eqn:heat conduction}, eq.~\ref{eqn:acoustic power}, eq.~\ref{eqn:drag force1}), Brownian motion energy coupling, thermal coupling, acoustic waves and friction are pressure dependent and are considerably reduced by maintaining low pressure. Therefore, the torsional pendulum is placed inside a vacuum chamber. The noise reduction results with smaller oscillations amplitude.
\par\noindent
The reduction of the amplitude enables to have active damping of the remaining noises. The damping system is built using two modulated light sources, each placed in front of a pendulum side mass. The modulated light is causing a small external radiation torque, which damps down the oscillations. Damping is made using a PID feedback loop, a tilt angle is measured, and an equivalent feedback external source is applied. With PID tuned correct the damped pendulum is at critical damping and oscillations become smaller. The measured error to $e(t) = \delta\theta(t)$ and the PID damping response $u(t)$ (eq.~\ref{eqn:PID response}) are both minimized. 
\par\noindent
Having a lower amplitude enlarges the SNR and the measurement sensitivity, which enables measuring smaller tilt angles. When amplitude is smaller than the fundamental limit (eq.~\ref{eqn:Brownian uncertainty 3}), the PID is effectively reducing the torsional pendulum's temperature. Introducing a new mass adds a constant new torque to the system $\tau_g$ (eq.~\ref{eqn:net_gravitation_torque_approx}), accordingly the average tilt angle becomes $\overline{\theta}_g$ (eq.~\ref{eqn:theta average}). The angle is proportional to the square of the period's length, the torsional pendulum's length and the mass are dropped out, indicating that short and light weight sensors work well as larger ones, as long as their periods are the same. 
\par\noindent
Thus, the PID measured error  changes to $e(t) = \delta\theta(t) + \theta_g(t)$, the PID is compensating for the new error with mainly a proportional gain $K_P$, and the response $u[e(t)]$ becomes mainly an inverse torque:
\begin{equation}
\tau_{PID} = u[e(t)] \approx K_P\theta_g(t) 
\label{eqn:PID_response}
\end{equation}
Assuming the mass is oscillating at a low frequency, when integrating over short periods the angle is constant in time and the average gravitational angle is given by:
\begin{equation}
\overline{\theta}_g = \frac{\int u[e(t)] dt}{\int K_P dt}
\label{eqn:PID_measurement_eqn}
\end{equation}
\par\noindent
According to eq.~\ref{eqn:theta average_2}, by accurate angle measurement, the gravitational field could be estimated. According to eq.~\ref{eqn:pid_gravitation_torque}, the measurement system allows fast response. 

\end{document}

