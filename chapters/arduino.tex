\providecommand{\main}{..}
\documentclass[\main/master.tex]{subfiles}
\begin{document}
\chapter{arduino}\label{chp:example-1}
\doublespacing
Arduino is an open-source microcontroller project used for building low cost and simple digital devices and circuits. Each microcontroller contains a microprocessor, controller, serial communication interface and is equipped with digital and analog input/output pins. The microcontrollers are controlled using C and C++ programming languages, and could be operated as standalone or connected to the computer through serial communication. 
\par
There are multiple Arduino board models, we would focus on Arduino Mega 2560. The Arduino Mega 2560 is based on the ATmega2560, an Atmel 8-bit AVR microcontroller. Also the board has 54 digital input/output pins of which 15 could be used as PWM outputs and a 16 MHz crystal oscillator (clock). In reality the arduino doesn't have analog output, to modulate the output Pulse Width Modulation technique is used.
\par
Digital control is used to create a square wave, using the 16 MHz clock signal switching on off diffrent voltages are simulates. The available voltafes are between the full Vcc of the board and off (5-0V), and t


\iffalse
\color{blue}
\par


 Digital control is used to create a square wave, a signal switched between on and off. This on-off pattern can simulate voltages in between the full Vcc of the board (e.g., 5 V on Uno, 3.3 V on a MKR board) and off (0 Volts) by changing the portion of the time the signal spends on versus the time that the signal spends off. The duration of "on time" is called the pulse width. To get varying analog values, you change, or modulate, that pulse width. If you repeat this on-off pattern fast enough with an LED for example, the result is as if the signal is a steady voltage between 0 and Vcc controlling the brightness of the LED.

In the graphic below, the green lines represent a regular time period. This duration or period is the inverse of the PWM frequency. In other words, with Arduino's PWM frequency at about 500Hz, the green lines would measure 2 milliseconds each. A call to analogWrite() is on a scale of 0 - 255, such that analogWrite(255) requests a 100% duty cycle (always on), and analogWrite(127) is a 50% duty cycle (on half the time) for example.



\fi
\end{document}