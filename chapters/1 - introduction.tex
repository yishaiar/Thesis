\providecommand{\main}{..}
\documentclass[\main/master.tex]{subfiles}



\begin{document}
\titleformat{\chapter}{}{}{0em}{\bf\LARGE}


\titleformat{\chapter}[display]
{\normalfont\huge\bfseries}{\chaptertitlename~\thechapter}{20pt}{\Huge}
\titlespacing*{\chapter}{0pt}{50pt}{40pt}
\titlespacing*{name=\chapter,numberless}{0pt}{-30pt}{10pt}

\chapter{Introduction}\label{chapter:Introduction}


The gravitational field has a very unique nature compared to other fields, the field is hard to mask and can not be shielded. The gravitational force acts along the intersecting line, proportional to the product of the masses and inverse to the square of the distance between masses. Due to the prevalence of stars with large masses, gravity is a dominating force in nature. Although, since Newton's gravitational constant $G$, is very small, the force is hard to measure in experiments. 
\par\noindent
Gravimetry, relative and absolute, is used widely to exam matter properties or measure gravitational fields. Precision gravimetry \cite{Wahr04,Bingham10,Bell98,Leeuwen00,Diorio03,Romaides01,Peters01,Luther82,Kuroda95,Karagioz96,Bagley97,Gundlach00,Quinn01,Armstrong03,Kleinevoss99,Parks10,Peters99,Mcguirk02,Dimopoulos07,Lamporesi08,Sorrentino10,Rosi14,Goodkind99} is used in oil and gas exploration \cite{Bell98}, mining \cite{Leeuwen00}, mapping earth's local gravity \cite{Wahr04,Bingham10} and temporal geological shifts. It is also used in accurate determination of Newton's gravitational constant \cite{Luther82, Kuroda95, Karagioz96, Bagley97, Gundlach00, Quinn01, Armstrong03, Kleinevoss99, Parks10, Peters99, Mcguirk02, Dimopoulos07, Lamporesi08, Sorrentino10, Rosi14}. 
\par\noindent
Most gravimetric methods rely on the Cavendish experiment, first performed in the 17th century. These measuring methods use a pendulum, where the sensor's tilt angle is caused by the gravitational field. The measurement is of the sensor's spectral density, and
have precision and accuracy limitations, especially at the low frequencies. 
\par\noindent
Significant effort has been undertaken to improve the measurement accuracy, as demonstrated with the torsion pendulum \cite{Luther82,Kuroda95,Karagioz96,Bagley97,Gundlach00,Quinn01,Armstrong03}, simple pendulum \cite {Kleinevoss99,Parks10} and atomic interferometry \cite{Lamporesi08,Sorrentino10,Rosi14}.
\iffalse
%Precision of the order of 1 $\mu$g to 1 nano-g is often used for mapping geological variations $(1 g = [9.8]m s^{-2} )$ .

%Both relative and absolute measurements are employed. 
%\par
\fi
\par\noindent
A standard in the industry for absolute gravimetry measurement is measuring interference fringe changes due to the free-fall of a corner cube in one arm of a Mach-Zehnder interferometer with a sensitivity of 100 nano-g $Hz^{-1/2}$. Where the measurement of spectral density $[\frac{g}{\sqrt{Hz}}]$, after integration results with a measurement $[g]$.
\par\noindent
Another competing gravimetric technology employs atomic interferometry achieving a resolution of 100 pico-g after two days of integration \cite{Peters01}. The field standard is a superconducting sphere suspended in the field of a superconducting coil achieving 3 pico-g resolution after one month integration and 1 pico-g  after one year \cite{Goodkind99}. 
The most sensitive device to date is Kasevich's 10 m atom interferometer which achieves 500 femto-g after one hour of integration \cite{PhysRevA.91.033629,kasevich2014prospects}. 
\par\noindent
In this work we propose a new inexpensive gravitational sensor, with high accuracy, using a low frequency pendulum. The sensor is based on a torsional pendulum with active thermal noise removal using radiation pressure. The sensor is a relative gravity sensor with an entirely optical readout, able to sense changes in gravitational fields around it. 
\iffalse
%Before we examine this analogy in profundity, let us briefly review the tunnelling ionization in
%strong fields theory in section 2.1. Then, in section 2.2, we will present the development of the
%above mentioned analogy and its principles. Next, in section 3, we will present the thesis aims,
%which is the application of the analogy, in two stages, first by numerical simulation and then
%experimental realization. Finally, in section 4, we describe the research methods and results of
%this both stages.
\fi
\par\noindent
The method suggested in this paper is used to achieve a better resolution at all of the spectral density frequencies, specifically at the lower frequencies. which are harder to measure, but are more useful. By using various methods of noise reduction we achieved higher SNR, which allows the measurement of significantly weaker gravity field signals.
\par\noindent
The method suggested in this work is simple and inexpensive, using off-the-shelf and inexpensive electronics and controlling them with an advanced algorithmic feedback control system.
 

\end{document}