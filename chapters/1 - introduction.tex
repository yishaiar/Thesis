\providecommand{\main}{..}
\documentclass[\main/master.tex]{subfiles}
\begin{document}
\chapter{Introduction}\label{chp:example-1}


Gravitational fields have a very unique nature compare to other fields. Gravitational fields could not be shielded and the force can not be compensated. The gravitational forces are inverse to the square of the distance between masses, weakening them. These properties makes Gravity a dominating force in nature but very hard to measure in experiments.
\\
Since the invention of the first modern gravimeter in 1936, gravimetry, relative and absolute, is used widely to exam matter properties or measure gravitational fields. Precision gravimetry \cite{Wahr04,Bingham10,Bell98,Leeuwen00,Diorio03,Romaides01,Peters01,Luther82,Kuroda95,Karagioz96,Bagley97,Gundlach00,Quinn01,Armstrong03,Kleinevoss99,Parks10,Peters99,Mcguirk02,Dimopoulos07,Lamporesi08,Sorrentino10,Rosi14,Goodkind99} is used in oil and gas exploration \cite{Bell98}, mining \cite{Leeuwen00}, mapping earth's local gravity \cite{Wahr04,Bingham10} and temporal geological shifts. It is also used in determining Newton's gravitational constant \cite{Luther82, Kuroda95, Karagioz96, Bagley97, Gundlach00, Quinn01, Armstrong03, Kleinevoss99, Parks10, Peters99, Mcguirk02, Dimopoulos07, Lamporesi08, Sorrentino10, Rosi14} and gravitational imaging systems. 
\\
Most gravimetric methods rely on the Cavendish experiment, first performed in the 17th century. These measuring methods use a pendulum and measure the spectral density of the gravitational field. These methods have precision and accuracy limitations especialy at low frequencies. Significant effort has been undertaken to improve the measurement accuracy, as demonstrated with the torsion pendulum \cite{Luther82,Kuroda95,Karagioz96,Bagley97,Gundlach00,Quinn01,Armstrong03}, simple pendulum \cite {Kleinevoss99,Parks10} and atomic interferometry \cite{Lamporesi08,Sorrentino10,Rosi14}.
\iffalse
%Precision of the order of 1 $\mu$g to 1 nano-g is often used for mapping geological variations $(1 g = [9.8]m s^{-2} )$ .

%Both relative and absolute measurements are employed. 
%\par
\fi
\\
A standard in the industry for absolute gravimetry is measuring interference fringes due to the free-fall of a corner cube in one arm of a Mach-Zehnder interferometer with a sensitivity of 100 nano-g $Hz^{-1/2}$. Another competing gravimetric technology employs atomic interferometry achieving a resolution of 100 pico-g after two days of integration \cite{Peters01}. 
The field standard is a superconducting sphere suspended in the field of a superconducting coil achieving 3 pico-g resolution after one month integration and 1 pico-g  after one year \cite{Goodkind99}. 
The most sensitive device to date is Kasevich's 10 m atom interferometer which achieves 500 femto-g after one hour of integration \cite{PhysRevA.91.033629,kasevich2014prospects}. 
\\
In this work we propose to advance a new gravitational sensor, with high accuracy, especialy at low frequencies. The sensor is based on a torsional pendulum with active noise removal using radiation pressure. The sensor is a relative gravity sensor with an entirely optical readout, able to sense changes in gravitational fields around it. 
\iffalse
%Before we examine this analogy in profundity, let us briefly review the tunneling ionization in
%strong fields theory in section 2.1. Then, in section 2.2, we will present the development of the
%above mentioned analogy and its principles. Next, in section 3, we will present the thesis aims,
%which is the application of the analogy, in two stages, first by numerical simulation and then
%experimental realization. Finally, in section 4, we describe the research methods and results of
%this both stages.
\fi
\\
The method suggested in this paper is used to achieve lower minimal signal measure-ment in all frequencies, specifically in lower frequencies, which are harder to measure,but are more useful. By using various methods of noise reduction we hope to achieve higher SNR, which would allow the measurement of significantly weaker gravity field signals.
\\
Our results promote the use of interferometric angle amplification techniques such as weak value measurements. 
\\
\iffalse
\begin{itemize}
	\item One
	\item Two
	\item Three
\end{itemize}
\fi
\end{document}