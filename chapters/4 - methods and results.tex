\providecommand{\main}{..}
\documentclass[\main/master.tex]{subfiles}
\begin{document}
\chapter{Methods and results}\label{chapter:Methods and results}

\section{System structure}
\subsection{Experiment setup}
\begin{figure}[htbp]
	\centering
	\fbox{\includegraphics[scale=0.15]{\main/images/4 - methods and results/setup.png}}
	\caption[The experiment setup]{The experiment setup}
	\label{fig:setup}
\end{figure}
\FloatBarrier
\par\noindent
The experiment setup, shown in fig.~\ref{fig:setup}, is composed by the torsion pendulum placed inside a vacuum chamber, tilt angle measurement system and a feedback system, all placed inside a seismic box. The angle is measured by a LASER beam tilted by the pendulum's front mirror and measured in a quadrant detector, the detector is connected to a computer by an oscilloscope through a signal amplifier. The tilt angle signal is reflected by a piezo driven mirror, which is tuned so the pendulum is balanced relative to the detector's center (the PID needs a reference for error calculations, otherwise it would calculate a wrong error and fail in damping). 
\par\noindent
The feedback system is composed by two LED light sources, modulated in real time by an Arduino micro-controller which is controlled by a PID feed-back loop in the computer. The LED light sources are placed in front of each of the viewports on the sides of the chamber, each coupled to a side mass of the pendulum. 
\par\noindent
When the desired vacuum is achieved, the vacuum engine is disconnected using the valve and turned off, and the seismic box is closed. When the oscillations settle down to a level which could be affected by the weak torques caused by radiation pressure (an amplitude caused mainly by environmental noises, as explained in previous chapters). The angle is read in real time by the PID algorithm, and an equivalent radiation-pressure torque is exerted by the LED's flux damping down the pendulum noises.
\begin{figure}[htbp]
	\centering
	\fbox{\includegraphics[scale=0.3]{\main/images/4 - methods and results/total_chamber.png}}
	\caption[Total chamber]{The system structure}
	\label{fig:Total chamber}
\end{figure}
\FloatBarrier
\par\noindent
The design of torsional pendulum and vacuum chamber, shown in fig.~\ref{fig:Total chamber}, was carried out using Solid Works. The design aims to minimize environment noises. As demonstrated previously (eq.~\ref{eqn:heat conduction}, eq.~\ref{eqn:total_kinetic}, eq.~\ref{eqn:acoustic_intensity}, eq.~\ref{eqn:drag force}), Brownian motion from the environment, thermal coupling, acoustic waves and friction are pressure dependent and are considerably reduced by maintaining low pressure. Therefore, the torsional pendulum is placed inside a vacuum chamber. Magnetic noise is reduced by choosing low magnetic permeability materials, avoiding capacitance and by having a Faraday Cage. 
\par\noindent
The vacuum chamber is placed inside seismic box, further reducing acoustic waves and magnetic noise. In order to maintain low pressure for long periods, system is designed to minimize the outgassing rate, by both choosing low outgassing materials and avoiding air pockets inside the devices. 
\subsection{Torsional pendulum design}
\begin{figure}[htbp]
	\centering
	\fbox{\includegraphics[scale=0.3]{\main/images/4 - methods and results/pendulum_front.png}}
	\caption[Torsional pendulum, front view]{Torsional pendulum, front view}
	\label{fig:pendulum front}
\end{figure}
\FloatBarrier
\par\noindent
The torsional pendulum design, shown in fig.~\ref{fig:pendulum front}, is composed of thin rod with length $2l$, two identical masses $m$ on the sides, a front mirror and a balancing mass. The torsional pendulum tilt angle is measured using a mirror in front of the pendulum. Due to an unbalanced center of mass, earth's gravity causes a downward torque $\tau_y$. 
%The design is maximizing the angle measurement sensitivity, while maintaining low magnetic noise and low vacuum pressure (low outgassing rate). 
%\subsubsection{Design}
\begin{figure}[htbp]
	\centering
	\fbox{\includegraphics[scale=1.2]{\main/images/4 - methods and results/pendulum_top.JPG}}
	\caption[The torsional pendulum, top view]{The torsional pendulum, top view}
	\label{fig:pendulum top}
\end{figure}
\FloatBarrier 
\par\noindent
As shown in fig.~\ref{fig:pendulum top}, the center of mass is balanced using a balancing mass with similar shape and weight as the mirror. The balancing mass and the mirror are connected to the pendulum by a spring $w$ from their center, and the downward torque is given by:
\begin{equation}
\tau_y = r_1\cdot m_1 g \cdot cos(0) - r_2\cdot m_2 g \cdot cos (0)\ = g( m_1 r_1  - m_2 (w-r_1) )  =0    \label{eqn:downward torque}
\end{equation}
Where $m_1$, $m_2$ are the masses of the mirror and the balancing mass, and $r_1$, $r_2$ are respectively their distance from the pendulum's center of mass. The spring allows accurate adjustment of the distance, given by: 
\begin{equation}
\frac{m_1}{m_2} = \frac{w-r_1}{r_1}   \label{eqn:downward torque cancelled}
\end{equation}


\subsubsection{Design constraints}
\par\noindent
The chosen pendulum dimensions are designed to achieve large angle sensitivity to torque caused by a gravity field (small string torsion coefficient $\kappa$, eq.~\ref{eqn:theta average}) while maintaining large angle sensitivity to mass (longer oscillation time period $T$, eq.~\ref{eqn:theta average}). For a torsional pendulum with moment of inertia $I$ (eq.~\ref{eqn:moment_inertia}) and string torsion coefficient $\kappa$ (eq.~\ref{eqn:torsion_coefficient_homogeneity}), the period $T$ is given by (eq.~\ref{eqn:undamped_omega}): 
\begin{equation}
T = 2\pi\sqrt{\frac{I}{\kappa}}= 2\pi\sqrt{\frac{2ml^2}{\kappa}} =  2\pi\sqrt{\frac{2ml^2}{\frac{G}{h} \frac{\pi d^4}{32}}}  \label{eqn:undamped_motion_equation_4}
\end{equation}
\par\noindent
As shown in eq.~\ref{eqn:undamped_motion_equation_4}, longer string $h$ with smaller diameter $d$ results with small string torsion coefficient while having a large period. Tungsten string is both vacuum compatible (material with low outgassing) and has high tensile strength, which allows holding the pendulum without tear inside the vacuum chamber while having small string diameter. 
\par\noindent
In order to minimize the magnetic noise in the measurements, the torsional pendulum is made out of stainless steel 316 (instead of stainless steel 304) with lower magnetic permeability \cite{SS316}, and the chosen front mirror is made fully of oxygen-free copper (OFC) instead of coated glass preventing capacitance. Having an OFC mirror, while not officially vacuum compatible, hopefully minimizes the vacuum outgassing.
\subsubsection{Technical information}
\begin{easylist}
& Tungsten string;
&& made of 99.95\% pure Tungsten
&& length $h$ of 249 mm
&& diameter $d$ of 0.08 mm
&& shear modulus $G$ of 130-160 [Gpa]\cite{tungsten}.
& Torsional beam;
&& made of Stainless steel 316
&& length $2l$ of 218 mm
% && width of 10 mm 
%& & weight of 90 gram 
&& identical side masses weight $m$ of 20.5 gram
& Mirror;
&& made of oxygen-free copper (OFC), gold coated
&& diameter of 1 inch
\end{easylist}
\begin{easylist}
& results;
&& expected;
&&& $I = 0.487\cdot10^{-3}[kg\cdot m^2]$
&&& $\kappa = 2.1\cdot10^{-6}[\frac{N\cdot m}{rad}] - 2.6\cdot10^{-6} [\frac{N\cdot m}{rad}]$
&&& $T = 96[s] - 86 [s]$
&& experimental;
&&& $\kappa = 2.7\cdot10^{-6}[\frac{N\cdot m}{rad}]$
&&& $T = 84[s]$
\end{easylist}



\subsection{Vacuum chamber design}

\begin{figure}[htbp]
	\centering
	\fbox{\includegraphics[scale=0.5]{\main/images/4 - methods and results/chamber_front.png}}
	\caption[Vacuum chamber, front view]{Vacuum chamber, front view}
	\label{fig:chamber front}
\end{figure}
\FloatBarrier

\par\noindent
The vacuum chamber, shown in fig.~\ref{fig:chamber front}, is composed of two cylindrical tubes placed one over the other with three view ports in front. The chamber is connected to a vacuum engine and gauge. The vacuum engine is connected to the chamber through a valve. Measurements are made when the valve is closed and the engine off, to prevent rotation noise.

\subsubsection{Chamber viewports}
\par\noindent
As shown in fig.~\ref{fig:Total chamber}, the central viewport is in front of the pendulum's front mirror. The central viewport has a 68.3mm view diameter with the mirror located 82 mm away, giving a measurement FOV of about $39^0$ degrees (instead of $90^0$ at interferometry). The two small viewports are in front of the pendulum's sensing masses to damp down the pendulum noises, and are connected to the chamber through light guides. The view ports are for 550-1100 nm, with about 98$\%$ power transmittance in the range. 

\subsubsection{Pendulum mount}
\begin{figure}[htbp]
	\centering
	\fbox{\includegraphics[scale=0.2]{\main/images/4 - methods and results/mount.png}}
	\caption[The pendulum mount]{The pendulum mount}
	\label{fig:mount}
\end{figure}
\FloatBarrier
\par\noindent
The upper tube of the chamber is soldered with an adjustable mount, shown in fig.~\ref{fig:mount}. The string of the torsional pendulum is held by the mount, enabling the adjustment of the string's length and placing the pendulum in front of the viewports accurately (shown in fig.~\ref{fig:Total chamber}). 
\subsubsection{Pressure increase}
\par\noindent
The main limitations to maintenance of low pressure are leakage $Q_L$ from the outside (eq.~\ref{eqn:leak rate}) and outgassing $Q_{des}$ inside the chamber (eq.~\ref{eqn:desorption rate}), increasing the pressure at a constant rate over time. The low pressure is maintained by having a pumping rate $Q_P$ given by:
\begin{equation}
Q_P \leq Q_L + Q_{des}  \label{eqn:vacuum_equilibrium}
\end{equation}
When the vacuum engine is working, the pumping rate is equal to the leak and outgassing and pressure is stable. In this experiment, due to the measurement sensitivity, the pumping must be off during measurement, causing pressure increase over time. Due to the pendulum's shape and size, the vacuum chamber is with large volume (high leak rate) and area (high outgassing rate).
\par\noindent
In order to minimize the pressure increase, the vacuum chamber is built using CF components which are designed for ultra high vacuum, and minimize the leak rate. Also, the torsional pendulum and vacuum chamber were baked-out to reduce outgassing, using resistive wire. The bake-out was carried out for a week at about $110 C^0$ achieving stable vacuum of about $2\cdot 10^{−4} [Hpa]$ after cooling down.

\subsection{Noise reduction}
As demonstrated previously (eq.~\ref{eqn:heat conduction}, eq.~\ref{eqn:total_kinetic}, eq.~\ref{eqn:acoustic_intensity}, eq.~\ref{eqn:drag force}), Brownian motion from the environment, thermal coupling, acoustic waves and friction are pressure dependent and are considerably reduced by maintaining low pressure.
I will add calculations..
\par\noindent
The measurement system is placed inside a seismic box which is a Faraday cage with 76 mm thickness, blocking magnetic fields with $f \ge 30 [Hz]$ from the environment. The seismic box is both reducing acoustic waves and magnetic noises from the environment.
\par\noindent
The vacuum chamber is a second cage blocking magnetic noise from the electronic system placed inside the seismic box. Vacuum chambers are made out of approximately 3 mm thick stainless steel, blocking magnetic fields with $f\ge 20 [KHz]$ while reducing magnetic noises from lower frequency fields.
\par\noindent

\section{Modulated light source}
\subsection{Radiation pressure force}
Two light sources with a given flux $\Theta_i$ causes a radiation-pressure force (eq.~\ref{eqn:radiation_force_power}), one on each mass of the torsional pendulum. As seen in eq.~\ref{eqn:net_gravitation_torque}, the difference between the two torques, adds an external net torque. Assuming that both light fields have the same coupling efficiency $\eta$ and are very close to being perpendicular to surface with negligible incidence angles $\alpha_1\approx\alpha_2\approx 0$, the radiation-pressure net torque is given by:  
\begin{equation}
\tau = l\cdot F_1 \cdot cos\alpha_1 - l\cdot F_2 \cdot cos\alpha_2\approx l(F_1 - F_2) \approx \frac{2l\eta}{{c}} (\Theta_1 -\Theta_2)  \label{eqn:radiation torque}
\end{equation}

\subsection{Arduino microcontroller}
Arduino is an inexpensive open-source microcontroller with a serial communication interface and a digital output without digital-to-analog converter (DAC). The Arduino Mega 2560 contains ATmega2560 8-bit controller, a $16 [MHz]$ crystal oscillator (clock) and 15 PWM outputs pins with digital output of $V_d = 5[V]$.
\subsubsection{Pulse Width Modulation}
\begin{figure}[htbp]
	\centering
	\fbox{\includegraphics[scale=0.5]{\main/images/4 - methods and results/duty_cycle.png}}
	\caption[The PWM analog voltage]{The PWM analog voltage}
	\label{fig:duty_cycle}
\end{figure}
\FloatBarrier
\par\noindent
The microcontroller can simulate analog output using Pulse Width Modulation (PWM). As shown in fig.~\ref{fig:duty_cycle}, the controller switches the output signal fast between the digital output on and off, generating a square wave with period $T$. Pulse width ($PW$) is the time duration in which the signal is on. With the clock, the controller is able to modulate $PW$, changing the ratio of time signal is on compared to off, which is the duty cycle $D$:
\begin{equation}
D(t) = \frac{PW(t)}{T}\cdot 100  \label{eqn:duty cycle}
\end{equation}
The duty cycle varies between 0 to 100\%, with resolution limited by the controller. The Arduino is modulating the duty cycle, resulting in modulation of the The analog voltage (the average voltage) $V_a$ given by: 
\begin{equation}
V_a(t) = \frac{ PW(t)\cdot V_d}{ T}  = \frac{V_d}{100}\cdot D(t)  = \frac{5}{100}\cdot D(t)  \label{eqn:pwm voltage}
\end{equation}
Since the Arduino clock is connected to all PWM pins, they are all in sync. When all PWM pins have the same duty cycle, they all have the same voltage, frequency and phase. Even though the pins output vary in time, they are constant voltage supplies compare to each other, allowing to connect $N$ pins in parallel and increase the output current $I_{total}$ given by: 
\begin{equation}
I_{total} = N\cdot I   \label{eqn:pwm current}
\end{equation}
The clock is having about 120 full periods of the square wave before changing to a new duty cycle value, limiting the voltage modulation frequency. Also, in order to be able to change the duty cycle, the clock frequency is limited by the controller bit-rate. The PWM frequency is given by:
\begin{equation}
f_{PWM} = \frac{1 }{120T}= \frac{1 }{120 \frac{8-bit }{16MHz}}  \approx 500[Hz]	    \label{eqn:pwm frequency}
\end{equation}
\subsection{Light emitting diode (LED)}
The light emitting diode (LED) is a semiconductor based light source with high power, and a long lifetime. Forward voltage is applied to a p-n junction, causing electron injection and recombination with holes. The recombination is releasing energy in form of spontaneous emission photons (incoherent light). Due to the electrons life time before recombination, the LED could be modulated up to $100MHz$.
\par\noindent
The LED opening angle (FOV) varies between 45 to 120 degrees, and the emitted light is incoherent in width meaning it's hard to focus it to a point (not diffraction limited) and incoherent in length causing wide band spectrum. The LED current $I$ is defined by the Shockley diode equation for p-n junctions:
\begin{equation}
I  = I_s (e^{V_l\cdot \frac{e}{n k_B T} }-1) = I_s (e^{\frac{V_d}{n V_T} } -1) \overset{V_d>>V_T}{\approx} I_s e^{\frac{V_d}{n V_T} }   \label{eqn:led current}
\end{equation}
Where $I_s$ is the saturation current of the diode, $V_l$ is the LED forward voltage, $n$ is the led emission coefficient, $e$ is the electron charge, and $V_T$ is the thermal voltage. The LED forward voltage is given by:
\begin{equation}
V_l \approx n V_T ln10 log_{10} (\frac{I}{I_s}) \label{eqn:led voltage}
\end{equation}
\subsection{LED setup}
Since the forward voltage $V_l$ varies as the logarithm of the current, it varies slowly, being approximately constant over wide current range, resulting with large changes in the LED current due to small changes in the circuit supply voltage $V$. Driving circuit composed of resistor $R$ in series with the LED stabilizes the current, given by:
\begin{equation}
I =\frac{V-V_l}{R} \label{eqn:led circuit}
\end{equation}
The circuit is composed of a blue LED with $V_l\approx 4.5V$, supply voltage with 6 parallel PWM Arduino pins (eq.~\ref{eqn:pwm current}, eq.~\ref{eqn:pwm voltage}) and resistor $R = 200\Omega$, the LED radiant flux $[W]$ is given by:
\begin{equation}
\Theta(t) = I_{total}\cdot V_l = NI\cdot V_l = N V_l \frac{V-V_l}{R}\cdot  =N V_l\frac{\frac{5}{100}\cdot D(t)-V_l}{R} \approx N V_l\frac{\frac{5}{100}\cdot D(t)}{R} = 0.007\cdot D(t)\label{eqn:led power}
\end{equation}
\par\noindent
In order to overcome the led uncoherent profile and large FOV, the light is coupled to a light guide, a pipe made of thin filaments causing internal reflections used  to illuminate small areas, regardless of the spectral characteristics of the light source. Light guide mainly depend on the entrance and exit cross section and length, making it ideal to overcome focusing problems. 
\par\noindent
Due to coupling efficiency and size difference between the output beam from the lightguide and the pendulum sides, the efficiency of flux hitting the pendulum is estimated to be $\eta = 0.7$ and the torque $\tau(t)_{LED}$ is given by:
\begin{equation}
\tau(t)_{LED}  =\frac{2l\eta}{{c}} (\Theta_1(t) -\Theta_2(t)) = 3.56^{-12}(D_1(t) -D_2(t)) 
\label{eqn:led torque}
\end{equation}
\par\noindent
The LED power could be approximated by linear approximation of the Arduino duty cycle $D(t)$, which varies between 0 to 100\% with 8bit resolution, generating torques up to $\tau = 7\cdot10^-{10} [N\cdot m]$ with modulation steps of $\Delta\tau = 1.4\cdot10^-{12} [N\cdot m]$. As shown in eq.~\ref{eqn:pwm frequency}), the LED could be modulated up to $500[Hz]$ resulting with fast modulated external force for the PID algorithm. Also, since the led power could be approximated using linear approximation, it allows to control the approximated power output by changing the PID gain values.  
\subsection{LASER}
Initially the modulated light sources were composed by a single high power LASER diode coupled in series into two acousto optic modulators (AOM), and coupled into the pendulum through optic fibers for mod cleaning. AOM can divide a single coherent light beam into two beams, with a proportional response to an input voltage, resulting with a controlled intensity ratio between the beams.
\par\noindent
The setup minimizes uncertainties of the light intensity since both light sources have a mutual source, causing a reduced torque uncertainty due to internal noises such as thermal and shot noise. The apparatus uncertainty due to the non linearity of the AOM response and LASER power fluctuations proved to be larger than the uncertainties of the intensity difference.   


\subsection{Damped oscillator}
The PID is damping a simple harmonic oscillator (eq.~\ref{eqn:undamped_motion_equation_solved}) with velocity given by:
\begin{equation}
\dot{\theta}_{max} = \omega_0\theta_{max} = \theta_{max}\cdot\frac{2\pi}{T}    \label{eqn:undamped_motion_equation_3}
\end{equation}
PID controller continuously calculates the oscillator's error value, damping it to zero. If the defined set point is set to zero, the error is the measured process variable $e(t) = \theta(t)$. The PID acts as friction, gradually working when the oscillations are at the maximum velocity to slow them down, and remove the torque energy:
\begin{equation}
\tau_{PID} = -\gamma\dot{e}(t) =    -\gamma\dot{\theta}(t)     \label{eqn:friction_torque_pid}
\end{equation}
Where $\gamma$ is the damping coefficient. The PID is controlling the LED radiation force, given in eq.~\ref{eqn:led torque}, where $D$ the duty cycle of each LED varies between 0 to 100%:
\begin{equation}
\tau_{PID} = \tau(t)_{LED} = 3.56^{-12}(D_1(t) -D_2(t))      \label{eqn:friction_torque_pid 2}
\end{equation}
The PID damping coefficient is given by: 
\begin{equation}
\gamma  =  \frac{max(\tau_{PID})}{\dot{\theta}_{max}} =       \frac{max(3.56^{-12}(D_1(t) -D_2(t)))}{\theta_{max}\cdot\frac{2\pi}{T}} \approx  \frac{1.1^{-10}T}{ \theta_{max} }     \label{eqn:damped_pid_motion_equation_2}
\end{equation}
As shown in eq.~\ref{eqn:damping_time} the damping time $\tau$ is given by:
\begin{equation}
\tau = \frac{1}{\xi\omega_0} =  \frac{2I}{\gamma} \approx \frac{2\cdot 0.487\cdot 10^{-3}}{\frac{1.1^{-10}T}{ \theta_{max} } } \approx 8.8\cdot 10^{6} \frac{\theta_{max}}{ T} \label{eqn:damping_time_pid}
\end{equation}
\par\noindent
The PID damping $\gamma$ and damping time $\tau$ depend on the initial oscillations angle. If the angle is too large compare to the external torque, the system is extremely underdamped (having a small $\xi$) and the PID affect is negligible, causing an infinite damping time, and pendulum would keep oscillating.

%The system is a damped oscillator with an external torque correction, shown in eq.~\ref{eqn:damped_motion_equation}.

\subsection{Control stability}
Using PID control does not guarantee optimal control or stability. The control system is aiming to achieve critical damping of the process. Well tuned control would a reach the desired set point fast and accurate, and also apply over time the necessary corrections to resist external forces trying to move variable from the set point.
\par\noindent
The controller response is its response to error. How much does the system overshoots a setpoint and the system oscillations. When controller gains are too high, instead of critical damping there is overdamping causing overshoot, due to the high gain the overshoot response overshoots to the other side, causing the system to be driven.










\section{Modulated CW laser}


\section{Modulated LED}

















\section{pid operate (algorithm)}
\section{laser +aom +amp}
\doublespacing



\section{Overshoot}
Overshoot is when output signal or function exceeds the target value. The response signal is not accurate compare to target. In control theory there are two wanted conflicting properties; an accurate response (small overshoot), and small risetime (fast response). 
\par\noindent
Overshoot is usually measured in percentage overshoot (PO). For second order systems, such as damped oscillators PO is a function of the damping ratio $\xi$. 


\begin{equation}
PO = 100\cdot e ^{\frac{-\xi\pi}{\sqrt{1-\xi^2}}} = \frac{output-target}{target}   \label{eqn:percentage_overshoot}
\end{equation}
 
\section{Accuracy}

\subsection{Laser}


The laser is a high power coherent light source based on stimulated emission with narrow spectrum. Inside a cavity Electron is excited to a higher energy level, and forced by photon with the correct wavelength to be absorbed emitting coherent photon. The coherence allows focusing beam to a tight spot, and being collimated over long distances. Continuous wave (CW) laser are lasing constant output power over time.



\par\noindent
Usually the power output is stable but has oscillations due to having several longitudinal modes causing nano second scale oscillations, output power is steady when averaged over longer periods. Also, over long periods of time lasers have slight power oscillations due to temperature changes in the environment.
%The laser diode cavity face is rectangular, because of fabrication constraints. The rectangular face is causing cylindrical aberration, which   

\subsection{Acousto optic modulator}
Acousto Optic Modulator (AOM), uses the acousto-optic effect to diffract and shift the frequency of light using RF waves. An oscillating electric signal drives a piezoelectric transducer to vibrate causing RF waves, the transducer attached to a material. This causes sound waves in the material, and periodic index modulation causing a Bragg grating. Incoming light scatters off the grating and due to Bragg diffraction comes at Bragg angle.
\begin{equation}
\theta_b = sin(\theta_b)\approx \frac{\lambda}{2n\Lambda} \label{eqn:aom angle}
\end{equation} 
$\Lambda$ is the RF wavelength, $\lambda$ is the light source wavelength. 
\par\noindent
Since all parameters constant, modulation angle is constant, and possible modulation speed is nano seconds. Giving a stable fast modulation method for CW laser.

\end{document}