\providecommand{\main}{..}
\documentclass[\main/master.tex]{subfiles}
\begin{document}
\newspacing
\chapter{Summary and conclusion}\label{chapter:Summary and conclusion}
The purpose of this study was to explore possibilities of improving the Cavendish apparatus gravitational measurement sensitivity and integration time, using a damping process on the oscillator motion. This was achieved by reducing the noises coupled into a torsional pendulum using a vacuum chamber and an acoustic box. A PID feedback loop was implemented, achieving active damping of the oscillations. The PID optical setup was adjusted to the specific torsional pendulum characteristics, by evaluation of the model for the PID damping process. The actual noise coupled into the torsional pendulum, and the fundamental limitations of the PID damping ability were estimated from the measured results.
\par\noindent
The achieved damped amplitudes were found to be at the fundamental limit of the room temperature thermal energy quantum uncertainty (eq.~\ref{eqn:Brownian uncertainty 3}). When damping beyond that amplitude, the damping process can assumed to effectively cool down the torsional pendulum temperature and lower the fundamental limit of the measurement. 
\par\noindent
As shown in previous chapter, with environment at room temperature and medium vacuum the fundamental limitation to the damped amplitude is the Brownian motion quantum uncertainty. The damping allows to cool the torsional pendulum down to less than $T=0.1[K]$ at the Brownian limit.
\par\noindent
This research demonstrates that PID feedback damping might not be the best control algorithm, due to its nature of linear motion approximation (having constant tuning parameters) and the difficulties of maintaining the right tuning parameters. Both effects have an increased influence as the damped amplitude decreases, thus increasing the damping challenge.
\par\noindent
The next stage would include a non linear approximation of the torsional pendulum motion, such as implementing an artificial neural network, towards the goal of advancing prospects of damping beyond the room temperature uncertainty. 

\end{document}