\providecommand{\main}{..}
\documentclass[\main/master.tex]{subfiles}
\begin{document}
\newspacing
\chapter{Summary and conclusion}\label{chapter:Summary and conclusion}
The purpose of this study was to explore possibilities of improving the Cavendish apparatus gravitational measurement sensitivity and integration time. This was achieved by designing a torsional pendulum with a long time period, and reducing the coupled noises using a vacuum chamber and an acoustic box. A PID feedback loop was than implemented, achieving active real-time damping process of the oscillator's motion. The damping process controls a modulated tourqe using LED light sources. 
\par\noindent
By evaluation of the PID damping model, the characteristics of the setup used by the PID were adjusted to the requirements of the specific torsional pendulum. The achieved damped amplitudes were found to be at the fundamental limit of the room temperature thermal energy quantum uncertainty (eq.~\ref{eqn:Brownian uncertainty 3}). The actual noise coupled into the torsional pendulum, and the fundamental limitations of the PID damping ability were estimated from the measured results. The results suggest that the amplitude could be damped up to the Brownian motion uncertainty limit. When damping beyond the thermal energy quantum uncertainty, the damping process can assumed to effectively cool down the torsional pendulum temperature and lower the fundamental limit of the measurement. 
\par\noindent
The oscillator's motion is the gravitational measurement uncertainty, limiting the measurement sensitivity. The significance of the study contribution is that, with the right conditions, even at room temperature the fundamental uncertainty limitation is the Brownian motion quantum uncertainty. The damping process allows to cool the torsional pendulum down to the Brownian motion limit, which is less than $T=0.1[K]$, resulting with gravitational measurement sensitivity higher than the current state of the art. As shown previously, the  measurement sensitivity can be achieved even with short and light weight sensors, using off-the-shelf and inexpensive electronics.
\par\noindent
This research demonstrates that PID feedback damping might not be the best control algorithm, due to its nature of linear motion approximation (having constant tuning parameters) and the difficulties of maintaining the right tuning parameters. Both effects have an increased influence as the damped amplitude decreases, thus increasing the damping challenge. The next stage would include a non linear approximation of the torsional pendulum motion, such as implementing an artificial neural network, towards the goal of advancing prospects of damping beyond the room temperature uncertainty. 

\end{document}