\providecommand{\main}{..}
\documentclass[\main/master.tex]{subfiles}
\begin{document}
\chapter{Summary and conclusion}\label{chapter:Summary and conclusion}
The purpose of this study was to find if it is possible to improve gravitational measurement sensitivity and speed using a damping control loop on an oscillator's oscillations. The way to achieve this was by building a Cavendish apparatus using a torsional pendulum and reducing the noises coupled into the system inside using a vacuum chamber. We than designed a PID damping feedback loop, which actively damps down the oscillations. The PID optical setup was adjusted to the specific torsional pendulum characteristics by evaluation of the PID damping model. From the measured results we than estimating the actual noise coupled into the torsional pendulum, and the fundamental limitations of the PID damping ability.
\par\noindent
The achieved damped amplitudes were up to the fundamental limit of the uncertainty at room temperature (eq.~\ref{eqn:Brownian uncertainty 3}). When reaching beyond that amplitude, the damping is effectively cooling down the torsional pendulum temperature and lowering the fundamental limit of the measurement. 
\par\noindent
As shown in previous chapter, with environment at room temperature and medium vacuum the fundamental limitation to the damped amplitude is the Brownian motion quantum uncertainty. The damping allows to cool the torsional pendulum down to less than $T=0.1[K]$ at the Brownian limit.
\par\noindent
From the study we think that PID damping might not be the best control algorithm, due to its linear approximation nature of the motion and the difficulties of having the right tuning parameters. Both effects have an increased influence as the damped amplitude is getting smaller, thus making it harder and harder to damp.
\par\noindent
The next stage would include a non linear approximation of the torsional pendulum motion, such as artificial neural network. Hopefully overcoming the difficulties, and allowing to damp down beyond the room temperature uncertainty. 



\end{document}