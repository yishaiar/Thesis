\providecommand{\main}{..}
\documentclass[\main/master.tex]{subfiles}
\begin{document}
\chapter{summary and conclusion}\label{chp:summary and conclusion}
The purpose of this study was to find if it is possible to improve gravitational measurement sensitivity by damping down the torsional pendulum oscillations. The way to achieve this was by building a Cavendish apparatus using a torsional pendulum inside a vacuum chamber, and designing a PID damping feedback loop, adjusted to the torsional pendulum characteristics. From the measured results we than estimating the actual noise coupled into the torsional pendulum, and the fundamental limitations of the PID damping ability.
\par\noindent
The observed damping amplitudes were up to the fundamental limit of the measurement (eq.~\ref{eqn:Brownian uncertainty 3}). As shown in previous chapter, with environment at room temperature and medium vacuum ($P=10^{-2}[Pa]$) the fundamental limitation to the PID damping is the Brownian motion quantum uncertainty. The fundamental oscillations amplitude limit is $\theta_{max}^{PID}\approx 2\cdot 10^{-14}$. When reaching that amplitude, the PID damping is effectively cooling down the torsional pendulum temperature and lowering the fundamental limit of the measurement (eq.~\ref{eqn:Brownian uncertainty 3}). The damping allows to cool the torsional pendulum down to less than $T=0.1[K]$.
\par\noindent
From the study we think that PID damping might not be the best control algorithm, due to its linear approximation nature. The next stage would include a non linear approximation of the torsional pendulum motion, such as artificial neural network. 


field physics. The way to achieve this was an analogy between the atom in a strong field prob-
lem and the problem of wave propagation in a curved waveguide. Within the framework of this

analogy there is a similarity between the adiabatic tunneling ionization and bending losses in
in such a waveguide. With this similarity, the fabrication of a curved waveguide allows direct
measurement of the wave function at any time, including the period when the electron is in the

’tunneling’ region. This waveguide fabrication was done in two ways, the first, used a finite ele-
ment numerical simulation (Comsol Multiphysics). In the simulation, the full Helmholtz equation

for the electromagnetic wave propagating through the curved waveguide is solved numerically
and demonstrates the feasibility of such a curved waveguide simulator. The simulation results
showed how an optical beam could mimic electron trajectories, including tunneling, free space
and re-collision. These results supported the feasibility of using this analogy. Nevertheless,
due to the limitations of the computer cause the accuracy analogy to be lost. Therefore the use
of the second method, producing a physical curved waveguide and measurement of its light
losses, was obvious. To obtain a step index waveguide, which the analogy uses, this physical
curved waveguide fabricated as a rib waveguide and wave mode calculated using the effective
index method. Several rib waveguide parameters were altered, during the experiment, due to

several undesirable results. The first undesirable result, inability to couple light that would prop-
agate along the waveguide. Another difficulty we encountered was in observing the leakage in

the curves, once we were able to couple light into the waveguide. Even once we were able to

overcome these difficulties, the observed results were not consistent with our theory. Unfortu-
nately, the initial leakage locations was not analogous to the theory of atom in a strong field. The

28

hypothesis of this discrepancy was that the roughness of the rib waveguide facet was not taken
into account. This roughness can cause additional losses beyond to the bending loss. This
hypothesis is not verified due to technical difficulties in measuring the waveguide roughness.
Another hypothesis is the possibility that the width of the waveguide mode expands so much
that it reaches the channels located at the sides of the waveguide, resulting in the light being
lost and leaking out. In this case, the way to improve the results is simple, and in the future it
will be wise to design a mask where this channel will be placed away from the waveguide, and
it may be worthwhile to remove them entirely. Such a design would allow light to leak out of the
waveguide according to the losses in optical waveguides theory, and hopefully, this leak might
fit the analogy.


\end{document}