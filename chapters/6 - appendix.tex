\providecommand{\main}{..}
\documentclass[\main/master.tex]{subfiles}
\begin{document}
\newspacing
\chapter{Appendix}\label{chapter:Appendix}
\iffalse
the noise is driving at the resonance frequency (noise profile irrelevant). 
\begin{equation}
f(\theta_{max},t) = \theta(t)  = \theta_g+\theta_{max} sin(\omega t)
\label{eqn:rms}
\end{equation}
after time integration:
\begin{equation}
\theta_{av}=E[\theta(t)]=\frac{1}{\pi}\int_0^{\pi}\theta_g+\theta_{max} sin(\omega t) dt=\theta_g+\frac{2 \theta_{max}}{\pi}  
\label{eqn:av}
\end{equation}
the average oscillates in time - we want to find variance $V$:
\begin{equation}
f(\theta_{av},t) =  \theta_{av} sin(\omega t) =\phi(t)
\label{eqn:rms}
\end{equation}

without measuring gravity $\theta_g$:
\begin{equation}
E[\phi(t)^2]=\frac{1}{\pi}\int_0^{\pi}\theta_{av}^2 sin^2(\omega t) dt = \frac{\theta_{av}^2}{2} =  \frac{\frac{(2 \theta_{max}}{\pi})^2}{2} = \frac{2\cdot \theta_{max}^2}{\pi^2} 
\label{eqn:av}
\end{equation}

\begin{equation}
E^2[\phi(t)]=(\frac{1}{\pi}\int_0^{\pi}\theta_{av} sin(\omega t) dt)^2 =(\frac{2 \theta_{av}}{\pi})^2  = \frac{8 E[\phi(t)^2]}{\pi^2}
\label{eqn:av}
\end{equation}


\begin{equation}
V[\phi(t)] = E[\phi(t)^2] - E^2[\phi(t)] = 0.189E [\phi(t)^2] = 3.8\cdot 10^{-2} \theta_{max}^2
\end{equation}



\begin{equation}
E[\phi(t)^2]=\frac{1}{\pi}\int_0^{\pi}\theta_{av}^2 sin^2(\omega t) dt = \frac{\theta_{av}^2}{2} =  \frac{(\theta_g+\frac{2 \theta_{max}}{\pi} )^2}{2} = \frac{\theta_g^2}{2} +\frac{2 \theta_g\theta_{max}}{\pi}+\frac{2\cdot \theta_{max}^2}{\pi^2} 
\label{eqn:av}
\end{equation}

\begin{equation}
E^2[\phi(t)]=(\frac{1}{\pi}\int_0^{\pi}\theta_{av} sin(\omega t) dt)^2 =(\frac{2 \theta_{av}}{\pi})^2  = \frac{8 E[\phi(t)^2]}{\pi^2}
\label{eqn:av}
\end{equation}


\begin{equation}
V[\phi(t)] = E[\phi(t)^2] - E^2[\phi(t)] =0.189 E [\phi(t)^2] = 0.189 (\frac{\theta_g^2}{2} +\frac{2 \theta_g\theta_{max}}{\pi}+\frac{2\cdot \theta_{max}^2}{\pi^2} )
\end{equation}
 since ($\theta_g<<\theta_{max}$):
\begin{equation}
V[\phi(t)] \approx 0.189\cdot \frac{2\cdot \theta_{max}^2}{\pi^2} =  3.8\cdot 10^{-2} \theta_{max}^2
\end{equation}
gravity measurement results with:
\begin{equation}
\overline{\theta}  \approx \frac{3GT^2cos\theta sin\theta}{4\pi^2 } \cdot \frac{M}{r_0^3} \pm 3.8\cdot 10^{-2} \theta_{max}^2 \label{eqn:theta average}
\end{equation} 
\fi


As shown in eq.~\ref{eqn:theta average} the gravimetric measurement is conducted by measuring the average equilibrium angle of the torsional pendulum. The measurement is limited by the quantum uncertainty due to the temperature thermal energy (eq.~\ref{eqn:Brownian uncertainty 3}). When the torsional pendulum is damped beyond the technical noises, the amplitude is limited by the temperature quantum uncertainty, the PID damping process is effectively cooling down the torsional pendulum temperature and thus reducing the measurement uncertainty. The gravimetric measurement is given by:
\begin{equation}
\frac{M}{r_0^3} \approx \frac{\overline{\theta}\pm \delta\theta}{\frac{3GT^2cos\theta sin\theta}{4\pi^2 }} \approx 5\cdot 10^{7}\cdot \overline{\theta}\pm 5\cdot 10^{7}\cdot \theta_{max}^{PID}
\label{eqn:measurement resolution}
\end{equation}
Assuming a measured mass $M$ at a given distance $r_0 = 1[m]$, the measurement sensitivity at room temperature is limited to $M\geq 5\cdot 10^{7}\cdot \theta_{max}^{PID}\approx 2[kg]$. The resolution limit is proportional to the damped amplitude, and thus lowered when reaching lower damped amplitudes.


\end{document}