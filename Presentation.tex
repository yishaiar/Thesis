
\PassOptionsToPackage{unicode}{hyperref}
\PassOptionsToPackage{naturalnames}{hyperref}
\documentclass{beamer} 
%\usepackage{babel}
%\usepackage[utf8]{inputenc}


%%% FONT SELECTION %%%%%%%%%%%%%%%%%
%%% we choose a sans font %%%%%%%%%%
\usepackage{kmath,kerkis} 
%\usepackage[default]{gfsneohellenic} 
%%%%%%%%%%%%%%%%%%%%%%%%%%%%%%%%%%%%

\usepackage{color}
\usepackage{amsmath}
\usepackage{amssymb}

\usepackage{epstopdf}
\usepackage{graphicx}
\graphicspath{{./images/presentation}}

%%
% load TEI-Pel - specific layout
\usepackage{TeiPel_En_Beamer_Layout}
\setTeipelLayout{draft,newlogo}% options: "draft", "newlogo"



% Thesis Info - first page 
	% title
		\title[Gravimetric Sensing Aided by Radiation Cooling]{Gravimetric Sensing Aided by Radiation Cooling}		
    %\author
		\author[Yishai Arieli]{Yishai Arieli}
	% supervisor	
		\supervisor{Supervisor}{Professor John Howell}
	% date
		\presentationDate{March, 2021}


\begin{document}

% typeset front slides
	\typesetFrontSlides

% Your Slides Start here:
\section{Introduction}



\subsection{Gravimetric measurements}
% frame 1
\begin{frame}{\hypertarget{frame:Gravitational field}{Gravitational field}}
	\begin{center}		
		\includegraphics[width=0.3\textwidth,keepaspectratio]{gravity.png}
    \end{center}
	\begin{itemize}
		\item The field vectors point towards the mass: $\overrightarrow{g}(M) = \frac{GM}{r^2}\hat{r}$ .
		\item Since the gravitational constant is small, the force is weak.
		\item Many gravimetric methods rely on the Cavendish experiment; using a pendulum.
		\item High sensitivity is a significant challenge at low frequencies.
		\end{itemize}
\end{frame}

% frame 2
\begin{frame}{Previous achievements}
	\begin{center}		
		\includegraphics[width=0.3\textwidth,keepaspectratio]{kasevich.png}
	\end{center}
	\begin{itemize}

		\item Interference fringe changes; sensitivity of 100 $\frac{nano-g}{\sqrt{Hz}}$.
		\item Atomic interferometry; sensitivity of of 100 pico-g after two days of integration.
		\item Atom interferometer; sensitivity of 500 femto-g after one hour of integration.
		
	\end{itemize}
\end{frame}

% frame 3
\begin{frame}{\hypertarget{frame:Harmonic oscillator}{Harmonic oscillator}}
	\begin{center}		
		\includegraphics[width=0.55\textwidth,keepaspectratio]{damp.png}
    \end{center}
	\begin{itemize}

		\item Torsional pendulum is an angular harmonic oscillator with a restoring torque, made of a mass hung by a string. 
		\item Undamped oscillator; harmonic oscillations without decay.
		\item Damped oscillator; there is a damping torque $\tau = -b\dot{\theta}$.
		\item Driven oscillator; there is also a time-dependent torque $\tau(t)$.
		
	\end{itemize}
	\hyperlink{frame:Damped oscillator}{>>}
\end{frame}


% frame 4
\begin{frame}{\hypertarget{frame:Cavendish apparatus}{Cavendish apparatus}}
	\begin{center}		
		\includegraphics[width=0.25\textwidth,keepaspectratio]{Cavendish apparatus.PNG}
    \end{center}
	\begin{itemize}
		\item Apparatus measures the gravitational torque of test mass.
		\item Mass attracts both masses $m$; creating a net torque $\tau_g$.
		\item The average equilibrium angle: $\overline{\theta}_g = \frac{\tau_g}{\kappa} \approx \frac{3GT^2cos\theta sin\theta}{4\pi^2 } \cdot \frac{M}{r_0^3}$.

	\end{itemize}
	\hyperlink{frame:Cavendish apparatus 1}{>>} 
\end{frame}


\subsection{Noises and damping}
% frame 1
\begin{frame}{\hypertarget{frame:Environmental noises}{Environmental noises}}
	\begin{center}		
		\includegraphics[width=0.4\textwidth,keepaspectratio]{total_chamber.png}
    \end{center}
	\begin{itemize}
		
		\item Environment gas cause pressure-dependent friction and noises (Brownian motion, acoustic waves, thermal flow).  
		\item Environmental black body radiation energy.
		\item Unknown Magnetic noise from the setup electronics.
		\item System placed inside a vacuum chamber and acoustic box. 
			
	\end{itemize}
	\hyperlink{frame:Environmental noises 1}{>>} 
\end{frame}


% frame 1
\begin{frame}{\hypertarget{frame:Measurement uncertainty}{Measurement uncertainty}}
	\begin{center}		
		\includegraphics[width=0.4\textwidth,keepaspectratio]{pendulum_front.png}
    \end{center}
	\begin{itemize}
		
		\item Angle is measured using a mirror and quadrant detector.
		\item Fundamental limits are: thermal uncertainty, shot noise. 
		\item Reducing the uncertainty enables higher sensitivity.
		\item Damping effectively lowers the temperature; $\delta\theta = \sqrt{\frac{k_B T}{\kappa}}$. 
		\item Resulting with lower temperature quantum uncertainty. 

		
	\end{itemize}
	\hyperlink{frame:Measurement uncertainty 1}{>>} 
\end{frame}

% frame 2
\begin{frame}{\hypertarget{frame:Proportional–integral–derivative (PID) controller}{Proportional–integral–derivative (PID) controller}}
	\begin{center}		
		\includegraphics[width=0.6\textwidth,keepaspectratio]{pid_diagram_powerpoint.jpg}
    \end{center}
	\begin{itemize}	
		\item PID controller is a feedback based control system, for time continuous control of a process (damped motion).
		\item The control is a weighted sum of proportional, integral and derivative response to error (P-I-D).
		\item Feedback applies an external correction to the process: $\tau(t)$.	
	\end{itemize}
	\hyperlink{frame:Proportional–integral–derivative (PID) controller 1}{>>} 
\end{frame}

% frame 3
\begin{frame}{\hypertarget{frame:Radiation tourqe}{Radiation tourqe}}
	\begin{center}		
		\includegraphics[width=0.2\textwidth,keepaspectratio]{radiation.PNG}
    \end{center}

	
	\begin{itemize}		
		
		\item Incident light beam causes force on the surface, due to momentum exchange with the electromagnetic field.
		\item Two light sources hitting the sides of the pendulum result with power-depended net tourqe.
		\item Power modulation is controlled by the PID feedback.
		
	\end{itemize}
	\hyperlink{frame:Radiation tourqe 1}{>>} 
\end{frame}
\begin{frame}{Damped oscillator}
	\begin{center}		
		\includegraphics[width=0.65\textwidth,keepaspectratio]{underdamp.png}
	\end{center}
	\begin{itemize}		
		\item PID mainly acts as friction, gradually slowing velocity.
		\item $\tau(t) =  \gamma\dot{\theta}(t) =  \gamma\frac{2\pi}{T} \theta( t) \rightarrow \gamma  =\frac{\tau_{max}}{\theta_{max}}\cdot \frac{ T}{2\pi} $
		\item When PID tuned correct; critical damping of process.
		\item Overshoot or slow response; result with driven oscillator.
		\item Torque small compared to amplitude; negligible damping. 
		
	\end{itemize}
\end{frame}

\begin{frame}{\hypertarget{frame:Gravimetric measurement}{Gravitational measurement}}
	\begin{center}		
		\includegraphics[width=0.3\textwidth,keepaspectratio]{Cavendish apparatus.PNG}
    \end{center}
	\begin{itemize}
		
		\item Introducing a new mass adds a constant torque $\tau_g$. 
		\item The PID is compensating for the new error; response becomes mainly an inverse torque: $\tau_{PID}(t) \approx K_P\theta_g(t)$. 
		\item When integrating over short periods: $\overline{\theta}_g =  \frac{\int \tau_{PID}(t) dt}{ K_P \Delta t} $. 	
		\item Lower uncertainty and short integration time.
	\end{itemize}
\end{frame}

\section{Methods and results}
\subsection{System structure}
% frame 1
\begin{frame}{Experiment setup}
	\begin{center}		
			\includegraphics[width=0.3\textwidth,keepaspectratio]{setup cropped.png}
	\end{center}
	\begin{itemize}

		\item Torsional pendulum is placed inside a vacuum chamber. 
		\item Tilt angle is measured by signal amplifier.
		\item The PID modulates two LED light sources in real-time.
		\item The LED are driven by Arduino controller; using PWM.
	
	\end{itemize}
	
\end{frame}

% frame 3
\begin{frame}{Vacuum chamber}
	\begin{center}		
		\includegraphics[width=0.4\textwidth,keepaspectratio]{chamber_front_names.PNG}
	\end{center}
	\begin{itemize}
		\item Chamber with transparent windows through light guides.
		\item Measurements conducted with vacuum engine off.
		\item Maintenance of low pressure; leakage, outgassing.
		\item Chosen materials minimize magnetic noise and outgassing.

		
	\end{itemize}	
\end{frame}


% frame 4
\begin{frame}{Torsional pendulum}
	\begin{center}		
		\includegraphics[width=0.4\textwidth,keepaspectratio]{pendulum_front_names.png}
	\end{center}
	\begin{itemize}
		\item Held by an adjustable mount; for accurate location. 
		\item Front mirror stabilized by a balancing mass.		
		\item Chosen dimensions aim achieving small $\kappa$ with large $T$.
		\item Measured: $T = 84[s]$, $\kappa = 2.7\cdot10^{-6}[\frac{N\cdot m}{rad}]$.
	\end{itemize}
	
\end{frame}

\begin{frame}{Previous laser setup}
	\begin{center}		
		\includegraphics[width=0.2\textwidth,keepaspectratio]{aom.png}
	\end{center}
	\begin{itemize}		
		\item Laser coupled in series into two acousto optic modulators; divided into two controlled beams.
		\item Range of $10000$ steps with frequency of $1 [Hz]$.
		\item Difference uncertainty minimized; mutual source.
		\item Output uncertainty large; AOM non linearity, fluctuations.
	\end{itemize}
\end{frame}

\begin{frame}{\hypertarget{frame:LED PWM circuit}{LED PWM circuit}}

	\begin{center}		
		\includegraphics[width=0.55\textwidth,keepaspectratio]{duty_cycle.png}
	\end{center}
	\begin{itemize}		
		\item LED coupled into a light guide (FOV of $45^0-120^0$).
		\item Linear response minimizes overshoot; $\tau(t)  \propto [D_1(t) -D_2(t)] $.
		\item Fast frequency minimizes phase delay; $f_{PWM} \approx 500[Hz]$.
		
		
	\end{itemize}
	\hyperlink{frame:LED PWM circuit 1}{>>} 
\end{frame}


\subsection{Results}
\begin{frame}{Damping}
	\begin{center}		
		\includegraphics[width=0.7\textwidth,keepaspectratio]{measured oscillation angle.png}
	\end{center}
	\begin{itemize}	
		
		\item Amplitude-depended damping; damping time of $3700[s]$ for $ 60 [\mu rad] \rightarrow 10[\mu rad] $, compared to $250[s]$ to reach $ 0.2[\mu rad]$.
						
	\end{itemize}
\end{frame}
\begin{frame}{Time integration}
	\begin{center}		
		\includegraphics[width=0.7\textwidth,keepaspectratio]{measured oscillation angle1.png}
	\end{center}
	\begin{itemize}	
		\item PID is able to damp up to an amplitude of $0.05[\mu rad]$.
		\item RMS amplitude kept by the PID over time of $ 0.2 [\mu rad]$.	
		
						
	\end{itemize}
\end{frame}
\begin{frame}{\hypertarget{frame:Quality factor}{Quality factor}}
	\begin{center}		
		\includegraphics[width=0.5\textwidth,keepaspectratio]{Q factor.png}
	\end{center}
	\begin{itemize}	
		\item Quality factor describes how damped the oscillator.
		\item Noise and amplitude depended: $Q =  \frac{\pi\kappa\theta_{max}^2}{T\cdot(\frac{\theta_{max}\pi\tau_{max}}{T} -p)} $.
		\item When $Q<0$; coupled noise larger than damping.
		\item Measurement coupled noise estimation: $p \rightarrow \frac{ \theta_{max}^{PID}\pi\tau_{max}}{T}$. 				
	\end{itemize}
	\hyperlink{frame:Quality factor 1}{>>} 
\end{frame}

\begin{frame}{\hypertarget{frame:Effective noise}{Effective noise}}
	\begin{center}		
		\includegraphics[width=0.3\textwidth,keepaspectratio]{random_motion1.jpg}
	\end{center}


	\begin{itemize}	
		\item Estimated noise smaller than calculated noises (Brownian motion, acoustic waves, thermal flow, black body).
		\item Only net collision power affects the motion uncertainty.
		\item Brownian motion and black body radiation are both treated as random particle collision processes ($\sqrt{N}$).
		\item Noise power composed only of net Brownian motion, net black body radiation and acoustic waves.  	 
					
	\end{itemize}
	\hyperlink{frame:Effective noise 1}{>>} 

\end{frame}


\begin{frame}{Resolution limit}
	
	\begin{itemize}	 
		\item Amplitude is limited by Brownian motion uncertainty.
		\item Allowing to damp down to $\theta_{max}^{PID}\approx 2\cdot 10^{-14}$.
		\item Achieved if damping lowers temperature; $\delta\theta = \sqrt{\frac{k_B T}{\kappa}}$.
		\item Resolution limit is proportional to the damped amplitude. 				
				
	\end{itemize}
\end{frame}


\section{Summary and conclusion}
\begin{frame}{Study achievements}
	\begin{itemize}
		
		
		\item Design a torsional pendulum with active motion damping. 
		\item Results suggest that the amplitude could be damped up to the Brownian motion uncertainty limit. 
		%\item When damping beyond the thermal quantum uncertainty, damping effectively cools down the pendulum temperature.
		
		
	\end{itemize}
\end{frame}

\begin{frame}{Study contribution}
	\begin{itemize}
		\item Improving sensitivity and integration time using active motion damping.

		\item Study contribution; with the right conditions, damping allows cooling up to the Brownian limit. 
		\item Resulting with higher gravimetric measurement sensitivity than current state of the art.
		\item PID not the best control algorithm; linear approximation. 
		\item Next stage would have non linear motion approximation.
		\item Goal of damping beyond room temperature uncertainty. 
	\end{itemize}
\end{frame}

\begin{frame}{Closing}
	\centering
	Thank you for your time. \\[12pt]
	
\end{frame}

\iffalse

\subsection[Basic Problem]{The basic problem that we have studed}

\begin{frame}{Slide Title \#1}
	\framesubtitle{Slide subtitle \#1}
	\begin{itemize}
		\item Use the \texttt{itemize} environment frequently.
		\pause
		\item Use short sentences and phrases.
		\pause
		\item In this presentation we use the \textbackslash{}\texttt{pause} macro.
	\end{itemize}
\end{frame}

\begin{frame}{Slide Title \#2}
	\begin{itemize}
		\item <1->You can define the order of appearance.
		\item <3->Like here.
		\item <2->This is the second item to appear.
	\end{itemize}
\end{frame}

\begin{frame}{Slide Title \#3}
	\begin{block}
		<1->{}
		\begin{itemize}
			\item Group without title.
			\item Appears for all slides.
		\end{itemize}
	\end{block}
	\begin{exampleblock}
		<2->{Group title}
		\begin{itemize}
			\item $e^{i\pi}=-1$.
			\item $e^{i\pi/2}=i$.
		\end{itemize}
	\end{exampleblock}
\end{frame}

%%
\subsection{Previous work}

\begin{frame}{Slide Title \#4}
	\begin{example}
		<1->First example. 
	\end{example}
	\begin{example}
		<2->Second example.
	\end{example}
\end{frame}

\begin{frame}{Slide Title \#5}
	\begin{center}
		Table example \\[12pt]
		\begin{tabular}{c||c|c|c|}
			& \textbf{col 1} & \textbf{col  2} & \textbf{col 3} \\
			\hline
			\hline
			\textbf{row 1} & 11 & 12 & 13 \\
			\hline
			\textbf{row 2} & 21 & 22 & 23 \\
		\end{tabular}
    \end{center}
\end{frame}

\begin{frame}{Slide Title \#6}
	\begin{center}
		Figure example \\[12pt]
		\includegraphics[width=0.35\textwidth,keepaspectratio]{LampFlowchart.png}
		\\
		\footnotesize(source: \textlatin{Wikipedia})
    \end{center}
\end{frame}

\begin{frame}{Slide Title \#7}
	\centering
	Math examples \\[12pt]
	\begin{equation}
        	B'=-\nabla \times E
	\end{equation}
	\begin{equation*}
        	E'=\nabla \times B - 4\pi j
	\end{equation*}
\end{frame}

%%%%
\section{Results / contribution}

%%
\subsection{Main results}

\begin{frame}{Summary}
   	\begin{alertblock}{Attention}
   		\textlatin{This is an important alert}
   	\end{alertblock}
\end{frame}

%
\subsection{Subsection title}

\begin{frame}{Summary}
	\begin{itemize}
		\item The \textcolor{red}{first main message} of our talk.
		\item The \textcolor{red}{second main message} of our talk.
		\item Maybe a \textcolor{red}{third message}, but ... no more.
	\end{itemize}
	\vskip0pt plus.5fill
	\begin{itemize}
		\item Conclusion.
	\end{itemize}
	\begin{itemize}
		\item Future work.
		\item Discussion.
	\end{itemize}
\end{frame}

\begin{frame}{References}
	\begin{thebibliography}{2}
		\beamertemplatebookbibitems
		\bibitem{Author1990}A.\ Author. \newblock\emph{Handbook of Everything}.\newblock
\textlatin{Some Press, \oldstylenums{1990}}.

		\beamertemplatearticlebibitems
		\bibitem{Someone2002}B.\ Author.\newblock On this and that\emph{.}
\newblock\emph{Journal on This and That}. 
\oldstylenums{2}(\oldstylenums{1}):\oldstylenums{50}--\oldstylenums{100}, 
\oldstylenums{2000}.
	\end{thebibliography}
\end{frame}
\fi





\section{Appendix}

% frame 2
\begin{frame}{\hypertarget{frame:Damped oscillator}{Damped oscillator}}

	\begin{center}		
		\includegraphics[width=0.55\textwidth,keepaspectratio]{damp.png}
    \end{center}
	\begin{itemize}	
		\item The damping ratio, $\xi$, determines the oscillator's motion: underdamped, critically damped or overdamped.
		\item Underdamped; oscillations amplitude decreases in time.
		\item Critically damped; exponential decay to equilibrium.
		\item Overdamped; exponential decay with longer damping time. 		
	\end{itemize}
	\hyperlink{frame:Harmonic oscillator}{<<} 
\end{frame}


% frame 3 - skip
\begin{frame}{\hypertarget{frame:Cavendish apparatus 1}{Cavendish apparatus}}
	\framesubtitle{Net torque}
	\begin{center}		
		\includegraphics[width=0.25\textwidth,keepaspectratio]{Cavendish apparatus.PNG}
    \end{center}
	\begin{itemize}
		\item The net gravitational torque is the sum inverse torques, one from each mass: $\tau_g = l \cdot F(r_1) \cdot cos\theta_1 - l \cdot F(r_2) \cdot cos\theta_2$
		\item If $l<<d,x,r_0$ the net torque can be approximated to: $\tau_g =  \frac{6l^2GmMxd} {r_0^5} = \frac{6l^2GmM sin\theta cos\theta}{r_0^3}$
		\item Where the tilt angle is $\theta_1 \approx \theta_2 \approx \theta$.
	\end{itemize}
	\hyperlink{frame:Cavendish apparatus}{<<} 
\end{frame}
% frame 4 - skip
\begin{frame}{Cavendish apparatus}
	\framesubtitle{Approximation}
	\begin{center}		
		\includegraphics[width=0.2\textwidth,keepaspectratio]{Cavendish apparatus.PNG}
    \end{center}
	\begin{itemize}
		\item The net gravitational torque: $\tau_g =  l d GmM(\frac{1}{r_1^3} - \frac{1}{r_2^3})$.
		\item Defining the function: $h(l) = \frac{1}{(d^2 +(x-l)^2)^{3/2}}$.
		\item If $l<<d,x,r_0$ approximation: $h(l)-h(-l)\approx h'(0)\cdot 2l = \frac{6lx}{r_0^5}$.
		\item the net torque is proportional to the difference: $\tau_g = l d GmM[h(l)-h(-l)]\approx \frac{6l^2GmMxd} {r_0^5}$.
	\end{itemize}
	\hyperlink{frame:Cavendish apparatus}{<<} 
\end{frame}
% frame 5 - skip
\begin{frame}{Cavendish apparatus}
	\framesubtitle{Simple harmonic oscillator}
	\begin{center}		
		\includegraphics[width=0.25\textwidth,keepaspectratio]{Cavendish apparatus.PNG}
    \end{center}
	\begin{itemize}
		\item Assuming no damping force (simple harmonic motion).
		\item At equilibrium, the torques cancel each other out $\tau_g =  \kappa\theta$.
		\item The average equilibrium angle: $\overline{\theta} = \frac{\tau_g}{\kappa} \approx \frac{3GT^2cos\theta sin\theta}{4\pi^2 } \cdot \frac{M}{r_0^3}$.
		
	\end{itemize}
	\hyperlink{frame:Cavendish apparatus}{<<} 

\end{frame}

% frame 2 - skip
\begin{frame}{\hypertarget{frame:Environmental noises 1}{Environmental noises}}
	\framesubtitle{Pressure and temperature dependent noises}
	\begin{itemize}
		\item An ideal gas has negligible inter-particle interactions.
		\item Brownian motion cause random collisions. 
		\item The kinetic energy of the gas particles: $ N<E_k> = \frac{3}{2} PV$.
		\item Acoustic waves are propagating with power of $10^{-22}AP^2$.
		\item Heat flow is the exchange of thermal energy between physical systems. The net power is: $p= A P c_v \Delta T \sqrt{\frac{M}{RT}} $.
		\item Net power from black body radiation is $p= A \sigma\epsilon[ T^4- T_0^4]$.
		
	\end{itemize}
	\hyperlink{frame:Environmental noises}{<<} 
\end{frame}


% frame 2 - skip
\begin{frame}{\hypertarget{frame:Environmental noises 2}{Environmental noises}}
	\framesubtitle{Friction}
	\begin{itemize}
		\item Friction, caused by drag force, decreases with pressure.		
		\item At higher pressures (turbulent flow); $F = -b(P)\cdot v^2 $.
		\item At lower pressures (gas with laminar flow), the drag force is given by Stokes' law: $F_{drag} =  -b(P)\cdot v$.	
		
	\end{itemize}
	\hyperlink{frame:Environmental noises}{<<} 
\end{frame}

% frame 5
\begin{frame}{\hypertarget{frame:Proportional–integral–derivative (PID) controller 1}{Proportional–integral–derivative (PID) controller}}
	\framesubtitle{PID response}
	\begin{center}		
		\includegraphics[width=0.6\textwidth,keepaspectratio]{pid_diagram_powerpoint.jpg}
    \end{center}
	\begin{itemize}
		\item The controller continuously calculates error from set point. 
		\item Each response is modulated by a tunable gain. 
		\item Output: $u(t) = K_P e(t)+K_I\int_{0}^{t}e(\tau)d\tau+K_D\frac{de(t)}{dt}$		
	\end{itemize}
	\hyperlink{frame:Proportional–integral–derivative (PID) controller}{<<} 
\end{frame}


\begin{frame}{\hypertarget{frame:Radiation tourqe 1}{Radiation tourqe}}
\framesubtitle{Radiation pressure}
	\begin{center}		
		\includegraphics[width=0.2\textwidth,keepaspectratio]{radiation.PNG}
    \end{center}

	
	\begin{itemize}		
		
		\item The force depend on the relative angle, surface reflectance and absorbance and the incident light power.
		\item With light perpendicular to a reflective surface $F  \approx\frac{2\eta\Theta}{{c}} $.
		
	\end{itemize}
	\hyperlink{frame:Radiation tourqe}{<<} 
\end{frame}

% frame 2 - skip
\begin{frame}{\hypertarget{frame:Measurement uncertainty 1}{Measurement uncertainty}}
	\begin{itemize}
		\framesubtitle{Fundamental limits}
		\item Basic energy level quantum uncertainty: $\delta\theta= \sqrt{\frac{\hslash\omega}{2\kappa}} \approx 10^{-16} [rad]$
		\item Quantum uncertainty due to thermal energy: $\delta\theta = \sqrt{\frac{k_B T}{\kappa}} \approx 4\cdot 10^{-8} [rad]$.
		\item Shot noise limit: $\delta\theta = \frac{1}{4\sqrt{2}\pi}\frac{\lambda}{L\sqrt{N}} \approx 10^{-14} [\frac{rad}{Hz}]$.
		
	\end{itemize}
	\hyperlink{frame:Measurement uncertainty}{<<}
\end{frame}


\begin{frame}{\hypertarget{frame:LED PWM circuit 1}{LED PWM circuit}}
	\framesubtitle{Arduino microcontroller}
	\begin{center}		
		\includegraphics[width=0.55\textwidth,keepaspectratio]{duty_cycle.png}
	\end{center}
	\begin{itemize}		
		\item Inexpensive 8-bit microcontroller with Pulse Width Modulation (PWM) analog outputs.
		\item Fast switching of digital signal results with analog signal.
		\item Analog signal depended on duty-cycle: $V(t) =  \frac{V_d}{100}\cdot D(t)$.
		\item Modulation limited by clock frequency: $f_{PWM} \approx 500[Hz]$.
	\end{itemize}
	\hyperlink{frame:LED PWM circuit}{<<} 
\end{frame}

\begin{frame}{\hypertarget{frame:Quality factor 1}{Quality factor}}
	\framesubtitle{Calculation}
	\begin{center}		
		\includegraphics[width=0.5\textwidth,keepaspectratio]{Q factor.png}
	\end{center}
	\begin{itemize}	
		\item Quality factor describes how damped the oscillator.
		\item Ratio of the energy stored to dissipated per cycle.
		\item Noise and amplitude depended: $Q =  \frac{\pi\kappa\theta_{max}^2}{T\cdot(\frac{\theta_{max}\pi\tau_{max}}{T} -p)} $.
					
	\end{itemize}
	\hyperlink{frame:Quality factor}{<<} 

\end{frame}


\begin{frame}{\hypertarget{frame:Effective noise 1}{Effective noise}}
	\framesubtitle{Calculated noises}
	\begin{itemize}	
		\item Thermal flow: $p=0.083[W]$.
		\item Black body radiation: $p=0.121[W]$.
		\item Brownian motion: $p=6.2\cdot 10^{-6}[W]$.
		\item Acoustic waves: $p=1.1\cdot 10^{-28}[W]$.
		\item New estimated noise: $p=1\cdot 10^{-19} - 7\cdot 10^{-19}[W]$.
					
	\end{itemize}
	\hyperlink{frame:Effective noise}{<<} 
\end{frame}
\end{document}